


POD TYMTO ZMAZAT - len inspiracia!!!

Analýza prezentovaná v tejto práci dokazuje, že aplikácia umelých neurónových sietí na marťanskú meteorologickú predpoveď
Premenné sú platným prístupom, ktorý je užitočný na doplnenie súčasných modelov, získanie prehľadu o atmosfére Marsu a analýzu údajov z Marsu
budúce meteorologické stanice na Marse.
Ukazujeme, že predpoveď pre tri študované miesta, t.j. VL1, VL2 a MSL, poskytuje výsledky s veľmi nízkymi chybami.
Ak je však sieť vyškolená pre jedno miesto určená na použitie s údajmi z inej oblasti, chyby sa zvýšia
dramaticky.
V našej práci sa snažíme vyvinúť minimálnu pracovnú ANN, ktorá by poskytovala realistické predpovede meteorologických údajov
vynaloženie minimálneho množstva času a počítačových zdrojov. Tento prístup je užitočný na vyplnenie medzier v súboroch údajov a tiež na štúdium
pôvod denných alebo hodinových výkyvov okolo predpovedaných hodnôt. Ako príklad sa zameriame na náhodné zmeny vzduchu
teplota cez noc a ukázať, že tlak nie je zodpovedný za tieto zmeny, pričom sa predpokladá dôvod týchto zmien
ako nočné turbulencie poháňané vetrom zo svahu. Ak by sa kapacita procesorov odoslaných na Mars naďalej zväčšovala
bude možné zahrnúť tieto algoritmy na spracovanie údajov in situ a optimalizovať merania tak, aby poskytovali čo najväčšie množstvo
informácií.