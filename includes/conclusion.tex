Cieľom práce bola analýza anonymizačných modulov, identifikačných prvkov prehliadača a vytvorenie anonymizačného modulu pre internetový prehliadač.\\
\indent Analýzou najpoužívanejších modulov a vlastností prehliadača, ktoré slúžia na identifikáciu používateľa, sme zistili aktuálny stav a funkcionalitu rozšírení, ktorými je možné anonymizovať prístup na internet. Väčšina týchto rozšírení modifikuje len časť vlastností prehliadača, ktoré sú odosielané na server, alebo úplne blokuje ich odosielanie. Nami vytvorené rozšírenie dokáže modifikovať väčšinu identifikačných prvkov rozšírenia, pričom dodržiava súvislosti medzi vlastnosťami (používateľský agent odosielaný v hlavičke dopytu je totožný s používateľským agentom zisťovaním pomocou JavaScript príkazu, súvislosť medzi šírkou a dĺžkou rozšírenia obrazovky). Dokáže blokovať údaje, ktoré sú posielané v otvorenej podobe na server a obsahujú informácie o identifikačných údajoch prehliadača, ktoré sa nedajú na úrovni rozšírení modifikovať. \\
\indent Testovanie rozšírenia nám overilo funkčnosť a správnosť implementácie. Rozšírenie dokáže buď vždy, alebo v časových intervaloch modifikovať väčšinu charakteristických prvkov prehliadača odsielaných na server, a tým zvyšuje anonymitu používateľa. 