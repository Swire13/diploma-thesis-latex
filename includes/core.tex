% define coloring in language JavaScript
\definecolor{lightgray}{rgb}{.9,.9,.9}
\definecolor{darkgray}{rgb}{.4,.4,.4}
\definecolor{purple}{rgb}{0.65, 0.12, 0.82}
\lstdefinelanguage{JavaScript}{
  keywords={break, case, catch, continue, debugger, default, delete, do, else, false, finally, for, function, if, in, instanceof, new, null, return, switch, this, throw, true, try, typeof, var, void, while, with},
  morecomment=[l]{//},
  morecomment=[s]{/*}{*/},
  morestring=[b]',
  morestring=[b]",
  ndkeywords={class, export, boolean, throw, implements, import, this},
  keywordstyle=\color{blue}\bfseries,
  ndkeywordstyle=\color{darkgray}\bfseries,
  identifierstyle=\color{black},
  commentstyle=\color{purple}\ttfamily,
  stringstyle=\color{red}\ttfamily,
  sensitive=true
}

\section{Motivácia}

\section{Teoretická časť}

\subsection{Počasie všeobecne}


\subsection{Počasie na Zemi}
\subsubsection{Atmosféra Zeme}
\paragraph{Atmosféra}je plynný obal Zeme, ktorý je pútaný ťiažovou silou k nej a zároveň s ňou aj rotuje. Udávaná horná hranica je 35-40 tisíc kilometrov, avšak samotná vrstva spojito prechádza priamo do kozmického priestoru, preto nie je možné udať presnú hodnotu hranice. Atmosféra je dynamický systém, v ktorom prebiehajú neustále procesy, ktoré sú avšak z dlhodobého hľadiska v relatívnej rovnováhe. Práve v atmosfére sa odohrávajú všetky javy počasia. Taktiež vďačíme práve atmosfére za podmienky vhodné na život na našej planéte, nakoľko zabezpečuje hneď niekoľko dôležitých aspektov:
\begin{itemize}
  \item zjemňuje výkyvy teplôt
  \item chráni organizmy pre slnečným a kozmickým žiarením
  \item chráni povrch pred dopadom menších kozmických telies
  \item zabezpečuje základné podmienky pre život
\end{itemize}
\paragraph{Vývoj atmosféry}na Zemi súvisí s geologcikými a geochemickými procesmi, taktiež aj s existenciou živých organizmov na našej planéte. Súčasná atmosféra Zeme je výsledkom evolúcie, keď pred 4.6 miliardami rokov obsahovala primárne ľahké plyny ako vodík a jeho zlúčeniny (napríklad metán), hélium alebo neón, dnesná atmosféra obsahuje primárne tie plyny, ktoré nereagovali s vodou - oxid uhličitý, dusík a podobne. Atmosféra postupom času chladla, vodná para kondenzovala, čo viedlo k tvorbe oceánov. Oxid uhličitý sa z atmosféry uvolňoval do oceánov a hornín, kde sa viazal v uhličitanoch. Až rozvoj živých organizmov, ktoré v procese fotosyntézy uvolňovali kyslík do ovzdušia, spôsobil jeho hromadenie v atmosfére Zeme. Vďaka tomu sa začala formovať ozónovrstva Zeme, ktorá začala chrániť Zem pred účinkami slnečného ultrafialového žiarenia. Pokles oxidu uhličitého v ovzduší a postupné navyšovanie kyslíka spôsobili pokles skleníkového efektu a markantné zníženie teploty Zeme na úroveň prijateľnú pre život na povrchu. Práve tento krok nenastal, keď sa pozeráme na inú planétu slnečnej sústavy - Venušu. Atmosféra, ktorú poznáme dnes, označujeme ako atmosféra III a registrujeme ju v časovom horizonte asi 400 miliónov rokov. 



\subsection{Mars}
\subsubsection{Všeobecne}
Mars je štvrtou a zároveň poslednou planétov vnútornej sústavy planét slnečnej sústavy. Okolo Slnka obieha vo vzdialenosti 227 miliónov a čas obehu okolo Slnka je 687 pozemských dní, čo predstavuje dĺžku dňa podobnú Zemskému, 24.6 hodín. Okolo Marsu obiehaju dva malé mesiace satelity - Deimos a Phobos, asteroidy, ktoré boli gravitačnou silou Marsu vtiahnuté na jeho obežnú dráhu. Oba satelity sú oproti satelitu Zeme - Mesiacu mnohonásobne menšie s priemermi 11 a 22 metrov. \cite{}
\subsubsection{Rok}
\subsubsection{Podmienky}
\subsubsection{Počasie}
\subsubsection{Obdobia}
\paragraph{Jarná rovnodennosť}
\paragraph{Letný slnovrat}
\paragraph{Jesenná rovnodennosť}
\paragraph{Zimný slnovrat}
\paragraph{Jesenná rovnodennosť}
\paragraph{Zimný slnovrat}
\paragraph{Začiatok sezóny púštnych búrok}
\paragraph{Koniec sezóny púštnych búrok}


\subsection{Možnosti riešenia predpovede}

\subsection{Existujúce riešenia}

\section{Praktická časť}

\subsection{Dáta}
\subsubsection{Dôležité parametre}
\subsubsection{Príprava dát}

\subsubsection{Implementácia}
\subsubsection{Typy neurónových sieti}
\subsubsection{Odskúšané modely}

\subsection{Výsledky}