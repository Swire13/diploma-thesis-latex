% define coloring in language JavaScript
\definecolor{lightgray}{rgb}{.9,.9,.9}
\definecolor{darkgray}{rgb}{.4,.4,.4}
\definecolor{purple}{rgb}{0.65, 0.12, 0.82}
\lstdefinelanguage{JavaScript}{
  keywords={break, case, catch, continue, debugger, default, delete, do, else, false, finally, for, function, if, in, instanceof, new, null, return, switch, this, throw, true, try, typeof, var, void, while, with},
  morecomment=[l]{//},
  morecomment=[s]{/*}{*/},
  morestring=[b]',
  morestring=[b]",
  ndkeywords={class, export, boolean, throw, implements, import, this},
  keywordstyle=\color{blue}\bfseries,
  ndkeywordstyle=\color{darkgray}\bfseries,
  identifierstyle=\color{black},
  commentstyle=\color{purple}\ttfamily,
  stringstyle=\color{red}\ttfamily,
  sensitive=true
}

\section{Motivácia}

\section{Teoretická časť}
V nasledujúcich podsekciách si postupne prejdeme základný prehľad teoretických informácií a poznatkov, ktoré sme počas procesu písania tejto práce získali. Uvedieme si základné informácie o tom, čo je počasie, aké počasie existuje na Zemi a aké počasie sa vyskytuje na planéte, ktorá sa v rámci slnečnej sústavy najviac podobá tej našej - Marse. Prejdeme si, akým spôsobom človek pristupoval k predpovedi počasia a ako sa v dnešnej dobe predpoveď vykonáva. Následne si zadefinujeme pojmy ako umelá inteligencia, strojové učenie a hlboké učenie, opíšeme, čo znamená slovo hyperparameter a ako ho vieme nastaviť. Na záver vysvetlíme pojmy výstupov a evaluácie, ktoré využívame pri praktickej časti a uvedieme existujúce riešenia, ktoré sú dostupné na internete a pojednávajú problematiku predpovede počasia na Marse.
\subsection{Počasie všeobecne}
Počasie, v najjednoduchšom ponímaní, predstavuje stav atmosféry na konkrétnej lokácii počas veľmi krátkeho časového intervalu. Zahŕňame do neho také atmosférické javy ako napríklad teplota, vlhkosť, zrážky, ktoré ďalej rozdelujeme podľa druhu a podľa množstva, tlak vzduchi, zloženie ovzdušia, vietor alebo oblačnosť. Často je tento pojem zamieňaný za pojem klíma, ktorý predstavuje informácie o podmienkach z dlhodobého časového horizontu, zväčša 30 rokov.

Počasie najčastejšie definujeme v oblasti zvanej troposféra, teda najnižšej vrstve atmosféry, ktorú budem opisovať v podsekcii \ref{atmosferaZeme}. Tieto atmosférické javy sú značne obmedzené práve na túto vrstvu, nakoľko sa v nej nachádzajú prakticky všetky zrážky a oblačnosť. Jednou z výnimiek sú triskové prúdy, ktoré ovplyvňujú priebeh atmosférického tlaku na hladine mora, čiže sa zapríčiňujú o vplyv na poveternostné podmienky. Počasie ovplyvňuje taktiež geografický reliéf Zeme, nakoľko za pomoci atmosférických javov môže dochádzať k zvetrávaniu a inej degradácii hornín \cite{meteo}. 

Vo všeobecnosti platí, že premenlivosť počasia sa z rôznou časťou sveta líši. Kým výraznejšie výkyvy počasia môžeme sledovať v stredných pásmach zemepisnej šírky, v tropických oblastiach sa počasie mení len sporadicky. Počasie je závislé aj od mnohých javov, ktoré nazývame anomálie. Ide o jav, ktorý pôsobí na počasie spôsobom, ktorý je veľmi špecifický a často výsledný efekt atmosferických javov je odlišný od očakávania. 

Okrem iného má počasie vplyv aj na osídlenie planéty. Čím lepšie podmienky počasia, tým lepšie podmienky pre pestovanie a celkové bývanie v danej lokalite. V prípade extrémov, akými sú prejavy typu tornádo, krupobitie, snehové búrky, tieto javy môžu ničiť nielen úrodu daného roka a teda spôsobovať problémy so zabezpečením základných surovín, môžu pustošiť aj ľudské obydlia a taktiež môžu ohroziť aj ľudské životy. Kým napríklad v prímorských oblastiach hrozí nebezpečenstvo silných búrok - cyklónov, na pevnine zase môže počasie narobiť problémy vo forme absencie zrážok v akomkoľvek skupenstve alebo silného vetra postupne silnejúcom naprieč dlhými rovinami.

Práve premenlivosť - prakticky hlavná vlastnosť počasia sa podpísala o to, že sa ľudia snažia vytvárať predpovedné modely počasia. Od historických pranostík, ktoré boli postavené na rôznych nevysvetlitelných konštruktoch až po dnešné predpovedanie počasia, ktoré sa vykonáva za pomoci superpočítačov, ktoré sú schopné vytvárať množstvo predpovedných modelov s určitou pravdepodobnosťou z dát získavaných pozemnými stanicami po celom svete a taktiež sieťou meteorologických satelitov. Ide o terabajty dát, ktoré je nutné spracovať za účelom vytvorenia modelu s čo najväčšou pravdepodobnosťou toho, ako bude počasie vyzerať v nasledujúcich dňoch naozaj. Aj tu platí, že kým v stredných pásmach zemepisnej šírky predpoveď počasia je menej presná, v tropických častiach sveta sa počasie dá predpovedať pomerne presne, nakoľko sa mení prevažne periodicky podľa fázy ročného obdobia \cite{meteo}.

\begin{figure}[!htbp]
  \centering
  \includegraphics[width=8cm]{img/numerical_prediction_model.png}
  \caption{Príklad numerického modelu zameraného na meranie vertikálneho pohybu atmosféry.}
  \label{numModel}
  \captionsource{Zdroj: }{https://www.ecmwf.int/en/about/media-centre/news/2015/course-probes-numerical-methods-weather-forecasting.}
\end{figure}

\subsection{Počasie na Zemi}
Zem je treťou planétou v rámci Slnečnej sústavy, zároveň sa nachádza v pásme vhodného pre vznik života. Doba obehu okolo svojej materskej hviezdy trvá 365 dní, rotácia okolo svojej osi trvá 23.93 hodín. Zem je dosiaľ jediné známe teleso s tekutou vodou vhodnou pre život. Až 70 percent zemského povrchu tvorí voda.

Z pohľadu počasia sa Zem vyznačuje komplexnými poveternostnými systémami. Vietor môže dosahovať rýchlosť až 240 km/h vo forme triskových prúdov. Počasie podlieha ešte zložitejším javom z dôvodu interakcie oceánov a iných veľkých vodných plôch ako napríklad morí. Mraky tvorí vodná para zmiešavaná s ľadom. Práve rovnováha týchto dvoch prvkov spôsobuje atmosférický jav nazývaný zrážky. Zrážky môžu vznikať primárne dvomi procesmi. V trópoch, kde sa ľad v oblakoch nevyskytuje ide o spájanie kvapiek vody v oblakoch. Po čase sa kvapky stanú dostatočne ťažké na to, aby opäť padli na povrch Zeme. Druhým spôsobom je interakcia vodnej pary a ľadu severne od oblasti trópov. V tomto prípade ľadové častice majú nižší tlak nasýtených pár ako voda. Postupne sa takéto ľadové kvapôčky spájajú až kým nie je častica natoľko ťažká, aby opäť dopadla na povrch. Vtedy padá na povrch Zeme buď ako dážď alebo vo forme krupobitia.

Výkyvy teplôt sa na Zemi pohybujú od veľmi veľkých v suchom podnebí až po minimálne vo vlhkých oblastiach. V poslednej dobe sa pozornosť upriamuje na skleníkový efekt, ktorý sme ľudskou činnosťou na Zemi dostali až do neželaných hodnôt. Tento efekt však už od počiatku Zeme stál za zrodom života, nakoľko správne množstvo sklenníkových plynov ako napríklad CO2 otepluje atmosféru Zeme na tú teplotu, ktorá sa označuje ako vhodná pre život. Problémom tohto efektu je aktuálne fakt, že teplota vzrastá do hodnôt, ktoré sú pre život neprijateľne z opačného hladiska ako zamŕzanie a teda práve topenie a zvyšovanie globálnej teploty aj lokálnej teploty do hodnôt, ktoré sú pre organizmus nebezpečné \cite{meteo}.

\subsubsection{Atmosféra Zeme}
\label{atmosferaZeme}
\paragraph{Atmosféra}je plynný obal Zeme, ktorý je pútaný ťiažovou silou k nej a zároveň s ňou aj rotuje. Udávaná horná hranica je 35-40 tisíc kilometrov, avšak samotná vrstva spojito prechádza priamo do kozmického priestoru, preto nie je možné udať presnú hodnotu hranice. Atmosféra je dynamický systém, v ktorom prebiehajú neustále procesy, ktoré sú avšak z dlhodobého hľadiska v relatívnej rovnováhe. Práve v atmosfére sa odohrávajú všetky javy počasia. Taktiež vďačíme práve atmosfére za podmienky vhodné na život na našej planéte, nakoľko zabezpečuje hneď niekoľko dôležitých aspektov:
\begin{itemize}
  \item zjemňuje výkyvy teplôt.
  \item chráni organizmy pre slnečným a kozmickým žiarením.
  \item chráni povrch pred dopadom menších kozmických telies.
  \item zabezpečuje základné podmienky pre život.
\end{itemize}

\paragraph{Vývoj atmosféry} na Zemi súvisí s geologcikými a geochemickými procesmi, taktiež aj s existenciou živých organizmov na našej planéte. Súčasná atmosféra Zeme je výsledkom evolúcie, keď pred 4.6 miliardami rokov obsahovala primárne ľahké plyny ako vodík a jeho zlúčeniny (napríklad metán), hélium alebo neón, dnesná atmosféra obsahuje primárne tie plyny, ktoré nereagovali s vodou - oxid uhličitý, dusík a podobne. Atmosféra postupom času chladla, vodná para kondenzovala, čo viedlo k tvorbe oceánov. Oxid uhličitý sa z atmosféry uvolňoval do oceánov a hornín, kde sa viazal v uhličitanoch. Až rozvoj živých organizmov, ktoré v procese fotosyntézy uvolňovali kyslík do ovzdušia, spôsobil jeho hromadenie v atmosfére Zeme. Vďaka tomu sa začala formovať ozónovrstva Zeme, ktorá začala chrániť Zem pred účinkami slnečného ultrafialového žiarenia. Pokles oxidu uhličitého v ovzduší a postupné navyšovanie kyslíka spôsobili pokles skleníkového efektu a markantné zníženie teploty Zeme na úroveň prijateľnú pre život na povrchu. Práve tento krok nenastal, keď sa pozeráme na inú planétu slnečnej sústavy - Venušu. Atmosféra, ktorú poznáme dnes, označujeme ako atmosféra III a registrujeme ju v časovom horizonte asi 400 miliónov rokov. 
\paragraph{Atmosféra Zeme} sa skladá z asi 78 percent dusíka, 21 percent kyslíka, 0,9 percenta argónu a 0,1 percenta iných plynov. Stopové množstvá oxidu uhličitého, metánu, vodnej pary a neónu sú niektoré z ďalších plynov, ktoré tvoria zvyšných 0,1 percenta. Dusík je molekula, ktorá sa do atmosféry dostáva procesom biologického rozkladu alebo sopečnou činnosťou. Zároveň je pre život nevyhnutný, nakoľko sa viaže v bielkovinách. Kyslík sa dostáva do atmosféry fotosyntézou. Je nevyhnutným prvkom, vďaka ktorému môžeme dýchať, taktiež je potrebný na rôzne oxidačné procesy. Argón je plyn, ktorý sa do atmosféry dostáva rádioaktívnym rozpadom draslíka v zemskej kôre. 
Ďalším plyn, ktorý je pre život na našej planéte nevyhnutný je ozón. Vzniká ionizáciou vzduchu napríklad pri búrke alebo taktiež fotochemicky - pôsobením slnečného UV žiarenia. Táto látka, napriek tomu, že je pre živé organizmy jedovatá, vo vyšších polohách atmosféry je pre život klúčová. V rozmedzí 10-50 km sa sústreďuje až 90 percent ozónu, preto tejto časti atmosféry hovoríme aj ozonosféra. 
\begin{figure}[!htbp]
  \centering
  \includegraphics[width=8cm]{img/ozone.jpg}
  \caption{Ozónová vrstva Zeme vyznačená bodkovaným polkruhom.}
  \label{numModel}
  \captionsource{Zdroj: }{https://www.dummies.com/article/academics-the-arts/science/environmental-science/what-causes-the-hole-in-the-ozone-layer-172790/}
\end{figure}
\newline Dôvod, prečo je tento plyn jedným zo základným predpokladom života je fakt, že ozón pohlcuje UV žiarenie Slnka, ktoré má škodlivé účinky na živé organizmy. V podsledných desaťročiach sa často spomína pojem ozónová diera, ktorá má za následok úbytok ozónu v atmosfére a to len na niektorých, presne lokalizovaných miestach, väčšinou nad územím severného a južného pólu. Tento jav je pozorovaný približne od 80. rokov minulého storočia a môže zaň najmä freón - uhľovodík, ktorý nahrádza atóm kyslíka za halové prvky ako napríklad flór alebo bróm. Po týchto zisteniach sa upustilo od používania freónov, vďaka čomu nastal proces revitalizácie ozónovej vrstvy, ktorý by mal byť ukončený v roku 2050.

\paragraph{Rozdelenie atmosféry} sa uplatňuje z rôznych hľadísk. Prvým, ktorý opíšem sa určuje podľa zmeny teploty voči výške. Merania dokázali, že teplota s výškou klesá, avšak v niektorých častiach atmosféry teplota narastá. Toto zistenie viedlo k záveru, že atmosféra sa skladá z viacerých vrstiev vzduchu, pričom hlavné vrstvy sú oddelené od seba tenkými prechodnými vrstvami.
\begin{itemize}
  \item Troposféra - výška hranice sa pohybuje od 8 km nad pólmi až po 17 km smerom k rovníku. V troposfére je sustredených približne 80 percent vzduchu a prakticky všetká vodná para, taktiež práve v tomto rozmedzí prebiehajú všetky prírodné javy, ktoré nazývame počasie. Dôležitým pojmom v rámci tejto vrstvy je vertikálny teplotný gradient, ktorý udáva pokles teploty vo vertikálnom smere. Priemerná hodnota sa pohybuje približne v hodnote 0.65 °C na 100 m vertikálnej výšky.
  \item Tropopauza - prechodná vrstva medzi troposférov a stratosférou. Hrúbka tejto hranice predstavuje rádovo niekoľko sto metrov, v záležitosti od podmienok ale može predstavovať až hodnotu do 3 km. Teplotný rozsah sa pohybuje v hodnotách -50 až -80 °C. 
  \item Stratosféra - vrstva oddelená tropopauzou od troposféry siahajúca po výšku 55 km od hladiny mora. V tejto vrstve, narozdiel od troposféry, teplota s narastajúcou vertikálnou výškou stúpa a môže dosiahnuť až 0 °C v najvyšších polohách. Dôvodom sú prebiehajúce fotochemické reakcie sposobené vplyvom slnečného UV žiarenia, konkrétne rozkladom molekuly ozónu, ktorého sa v tejto vrstve nachádza až 90 percent z celkového objemu v atmosfére, preto sa táto oblasť nazýva aj ozónosféra. Koniec tejto vrstvy ohraničuje prechodová vrstva - stratopauza. 
  \item Mezosféra - nachádzajúca sa v rozmedzí 50 - 80 km, pričom teplota s výškou klesá až do -80 °C, zaujímavosťou je, že teplota klesá viac v lete ako v zime. Vo stúpajúcom vertikálnom smere ju následne oddeľuje mezopauza.
  \item Termosféra - vrstva siahajúca až do výšky 300 km, pričom pre posledných 100 km je význačný vzostup teploty s výškou, ktorá na konci predstavuje hodnotu až 1000 °C. V tejto vrstve je vzduch plne ionizovaný, teda obsahuje výhradne elektricky nabité častice. Práve z tejto oblasti je možné pozorovať žiaru meteorov v atmosfére. Taktiež je to oblasť, v ktorej je možné pozorovať polárnu žiaru, nakoľko sú práve v tejto vrstve ideálne podmienky na vstup nabitých častíc zo Slnka. Oddelujúca vrstva - termopauza sa uvádza ako oblasť, kde ešte tento jav môžeme pozorovať. 
  \item Exosféra - finálna časť atmosféry, ktorá už nemá konečné ohraničenie, teda voľne pokračuje do otvoreného kozmu. 
\end{itemize}

\begin{figure}[!htbp]
  \centering
  \includegraphics[width=8cm]{img/atmosphere.png}
  \caption{Atmosféra Zeme v závislosti s vývojom teploty.}
  \label{numModel}
  \captionsource{Zdroj: }{https://mccord.cm.utexas.edu/chembook/page.php?chnum=2&sect=3}
\end{figure}

Ďalším spôsobom členenia je podľa homogenity vzduchu: 
\begin{itemize}
  \item Homosféra - vertikálny priestor až do výšky 80 km od zemského povrchu. Zaraďujeme sem časti od tropsféry až po hranicu mezopauzi. Charakteristikou je, že v nej nepriebieha obmena zastúpenia zložiek vzduchu.
  \item Heterosféra - s hranicou od mezopauzi ide o časť, v ktorej sa znižuje koncentrácia ľahkých plynov so vzdialenosťou pomalšie ako koncentrácia ťažkých plynov. To je dôvod, prečo vo výške viac ako 1000 km od Zeme prevláda vodík. V tejto oblasti sa taktiež pozoruje priamy vplyv slnečného a kozmického žiarenia, čo spôsobuje napríklad aj jav fotoionizácie, ktorý zapríčiňuje zvýšenú teplotu v týchto oblastiach.
\end{itemize}

Taktiež je možné členenie podľa interakcie so zemským povrchom. Toto členenie je dôležité najmä kvôli zisťovaniu denných meteorologických údajov, pričom rozpoznávame rozdelenie na dve časti:

\begin{itemize}
  \item Hraničná vrstva atmosféry - v tejto časti sa výrazne prejavuje vplyv povrchu Zeme na prejav meteorologických údajov. Ohraničená je výškou, nad ktorou je už prúdenie vzduchu ovplyvnené len tiažovou silou, prípadne rozložením tlaku. Hranica taktiež záleží od reliéfu zemského povrchu, nad ktorou sa nachádza, pri členitejšom reliéfe sa nachádza vyššie ako pri rovinatom teréne. Všeobecne sa ale výška udáva v hodnotách a približne 1500 m. Nad touto úrovňou už nie je teplota vzduchu výrazne závislá od vplyvov zemského povrchu.
  \item Voľná atmosféra - nachádzajúca sa nad hraničnou vrstvou atmosféry. V tejto časti atmosféry už nemajú fyzikálne deje ani meteorologické vplyvy prakticky žiadny efekt.
\end{itemize}

\subsection{Mars}
Mars je štvrtou a zároveň poslednou planétou vnútornej sústavy planét slnečnej sústavy. Okolo Slnka obieha vo vzdialenosti 227 miliónov kilometrov a čas obehu okolo Slnka je 687 pozemských dní, čo predstavuje dĺžku dňa podobnú našej - 24.6 hodín. Okolo Marsu obiehaju dva malé satelity - Deimos a Phobos, asteroidy, ktoré boli gravitačnou silou Marsu vtiahnuté na jeho obežnú dráhu. Oba satelity sú oproti satelitu Zeme - Mesiacu mnohonásobne menšie s priemermi 11 a 22 metrov. Teploty na Marse sú v priemere okolo -63 °C. Teploty sa však pohybujú od okolo -140 °C v zime na póloch až po 21 °C v nižších zemepisných šírkach v lete \cite{pocasie_mars}.

\subsubsection{Atmosféra Marsu}
\label{atmosferaMarsu}

Za posledné desaťročie bolo na povrch Marsu vypustených mnoho sond za účelom zistenia konzistencie povrchu. Ak by sme chceli prirovnať podmienky na Marse k tým na Zemi, najviac by sa povrch podobal púštným oblastiam. Povrch planéty je posiaty krátermi, ktoré boli vplyvom silného vetra na povrchu postupne uhladené a teda dnes vidíme len malé pozostatky. Práve vietor je jedným z primárnych atmosférických javov na povrchu planéty, nakoľko jeho pôsobením môžu nahromadené prvky podobné piesku vzniesť do ozvdušia a spôsobiť púštne búrky, ktoré sa v rámci roka dokonca na Marse periodicky opakujú. Následne môžeme občas dokonca aj z našej planéty za pomoci silnejšieho ďalekohľadu vidieť výraznejšie začervenanie planéty ako obvykle. Hornina, z ktorej je tento prach na povrchu Marsu zložený pozostáva najmä z danej červenkastej horniny, piesku a pôdy. V niektorých oblastiach Marsu sa vyskytujú miesta so zeleným zafarbením, avšak tento efekt nebol do dnešného dňa vysvetlený. Čo sa týka vody, známe sú ložiská vody pevného skupenstva na póloch planéty. Kvapalnú vodu sa nám do dnešného dňa nepodarilo objaviť napriek dôkazom erodovanej pôdy z plyvu práve vody kvapalného skupenstva. Taktiež sa predpokladá, že voda v kvapalnom skupenstve existuje aj pod povrchom Marsu, avšak vplyvom nízkeho atmosférického tlaku by sa voda vystupujúca na povrch ihneď pretvorila do plynného stavu. 
Atmosféra Marsu sa skladá predovšetkým z oxidu uhličitého. Avšak na rozdiel od Venuše je atmosféra Marsu veľmi tenká, vystavuje planétu bombardovaniu kozmickým žiarením a vytvára veľmi malý skleníkový efekt. Atmosférický tlak na povrchu Marsu je iba 1 až 2 percentá tlaku na Zemi. 
Tak ako na Zemi aj Mars má atmosféru a teda aj počasie. To, ako definujeme tieto dva pojmy na tejto planéte je však mimoriadne odlišné od toho, ako ich vnímame na našej planéte. Ako sme spomínali v sekcii \ref{atmosferaZeme}, Zemskú atmosféru tvorí asi 78 percent dusíka, 21 percent kyslíka, 0,9 percenta argónu a 0,1 percenta iných plynov. Taktiež obsahuje asi 1\% (backslash percent) vodnej pary. Atmosféra Marsu však pozostáva z 95\% (backslash percent) oxidu uhličitého, 3\% (backslash percent) dusíka, 1,6\% (backslash percent) argónu a obsahuje stopy kyslíka, oxidu uhoľnatého, prípadne vody. Vzduch na Marse je veľmi riedky a tlak v porovnaní je taktiež iba 1 až 2 percentá tlaku na Zemi. Pre predstavu, takýto tlak sa vyskytuje vo výške 45 kilometrov našej atmosféry. Tento tlak sa ešte mení s nadmorskou výškou, avšak stále pojednávame o veľmi malých, až zanedbateľných zmenách. Zaujímavosťou v rámci kolísania tlaku je však jeho periodické kolísanie v rámci sezóny. Je to z dôvodu vysokého množstva plynu CO2 v atmosfére, ktorý sa mení s ročnými obdobiami. Vyšší tlak sa vyskytuje počas južných letných mesiacov a najnižší počas severných letných mesiacov. Zmena je spôsobená rozdielom teplôt v danom období. Kým severné polárne zimy sú teplé a krátke, južné polárne zimy sú dlhšie a teplota klesá k nižším hodnotám. Práve južná polárna zima spôsobuje zamŕzanie plynu CO2 priamo v oblastiach južného pólu, čo spôsobuje pokles tohto plynu v atmosfére. Vtedy tlak na Marse klesá o približne 30 percent.
Cirkulácia atmosféry Marsiu je omnoho jednoduchšia ako na Zemi. V nižších zemepisných šírkach je dominantným javom pohyb Hadleyho buniek, ktoré predstavujú stúpajúci zohriaty vzduch v oblasti rovníka. Jedna časť prúdi na sever, druhá na juh do oblasti 30° zemepisnej šírky (do oboch smerov). Tam sa ochladzuje, klesá a prúdi opať k rovníku. V priamom toku zo severu na juh bráni rotácia planéty a reliéf Marsu. Vo vyšších zemepisných šírkach vznikajú polárne vzduchové masy, zároveň sa zo západu na východ tiahne séria oblastí vysokého a nízkeho tlaku. V týchto miestach v čase interakcii s Hadleyho bunkami môžu vznikach fronty počasia pretavujúce sa do búrok. Tie sú však v porovnaní s pozemskými omnoho pokojnejšie.
Veľký rozdiel Zeme a Marsu spočíva v tom, že Mars nemá žiadnu ozónovú vrstvu. Z toho vyplýva, že ultrafialové žiarenie zo Slnka nerušene dosahuje na povrch a teda škodlivo pôsobí na akékoľvek organické zlúčeniny na povrchu. Je to tiež dôvod, prečo atmosféra Marsu nemá žiadnu vrstvu zodpovedajúcej stratosfére Zeme, nakoľko by v takejto vrstve vysokonabité častice nemali s čím interagovať. 
Okrem vyššie spomínaných prašných búrok sa môžu vyskytovať na Marse aj oblaky zložené z ľadových častíc CO2, ktoré sa primárne sústreďujú v oblasti veľkých sopiek, kde sa najčastejšie tvoria vďaka dvíhaním častíc vetrom do vyšších oblastí, kde následne tieto častice kondenzujú. Taktiež sa môžeme stretnúť aj s takzvanými polárnymi kuklami - oblačnosti v polárnej oblasti, ktotér tvoria široké opary. Tento jav zachytilo už niekoľko pozemných roverov, ktoré zaznamenali sneženie počas chladných rán pred východom Slnka \cite{pocasie_mars}. 

\subsubsection{Ročné obdobia}
Teplota na Marse je oveľa nižšia ako na Zemi. Hlavné faktory, ktoré sa pod nízku teplotu podpisujú sú najmä vzdialenosť od Slnka a taktiež, ako sme spomínali vyššie, oveľa tenšia atmosféra tvorená prakticky výhradne oxidom uhličitým. Táto kombinácia faktorov robí z Marsu studený svet, kde môže teplota klesnúť až na hodnoty -120 °C, teda nižšie ako hodnoty namerané na našej planéte, kde sa podarilo doposiaľ namerať najnižšiu teplotu na úrovni -90 °C. V prípade najvyšších teplôt na Marse ide o hodnoty približne 20 °C. Táto teplota je výrazne nižšia v porovnaní s maximálnou nameranou teplotou na našej planéte, čo spôsobuje najmä tenká vrstva atmosféra, ktorá nie je schopná udržať tepelnú energiu. Aj preto sa priemerná hodnota teplôt na Marse stanovuje na úroveň -60 °C. Často je počas noci možné badať jav tvorenia mrazu na skalách, ktorý sa následne blížiacim úsvitom mení na paru, čím vzniká vlhkosť v ovzduší. Práve tento jav by v budúcnosti mohol prispieť k tomu, aby sa Mars stal obývateľnejším miestom. 
Podobne ako na Zemi, aj na Marse môžeme sledovať periodické striedanie štyroch ročných období, nakoľko sa planéta nakláňa okolo svojej osi. Tým, že excentricita planéty je výrazne väčšia ako Zeme, dĺžka ročných období je odlišná. Kým jar je na severnej pologuli najdlhším ročným obdobím s trvaním približne 7 pozemských mesiacov, leto a jesen zhodne trvajú približne 6 zemských mesiacov a zima ako najkratšie obdobie trvá približne 4 pozemské mesiace \cite{mars_meteo1}.
Každé ročné obdobie však trvá približne dvakrát dlhšie, pretože marťanský rok je približne dvakrát dlhší ako na Zemi. Mars obieha najbližšie k Slnku, keď je jeho južná pologuľa naklonená smerom k nemu, zatiaľ čo severná pologuľa je naklonená k Slnku, keď je od Slnka najďalej. Južné leto je preto oveľa teplejšie ako severné leto. Toto extra teplo prúdiace na južnú pologuľu spôsobuje väčšie turbulencie a poháňa silnejší vietor, ktorý vyvoláva najväčšie búrky. 
Pre jar na Marse je príznačné topenie oxidu uhličitého nahromadeného v polárnych častiach planéty. Táto pokrývka je sezónna a teda s rôznym ročným obdobím nadobúda oxid uhličitý uložený na týchto miestach rôzne skupenstvo.
\begin{figure}[!htbp]
  \centering
  \includegraphics[width=8cm]{img/co2.jpg}
  \caption{Oxid uhličitý v pevnom skupenstve na povrchu Marsu.}
  \captionsource{Zdroj: }https://mars.nasa.gov/imgs/2016/07/MRO-hirise-Spiders\_dry-ice-mars-thmfeat.jpg}
  \label{co2}
\end{figure}
\newline Počas jari severná ľadová pokrývka sublimuje, pričom sa oxid uhličitý mení z pevného skupenstva priamo na plyn. To, koľko plynu sublimuje záleží od slnečného svetla, ktoré dopadá na danú oblasť. S pokračujúcou jarou všetok ľad v oblasti sublimuje, pričom plyn v atmosfére viaže na seba aj prachové častice. Akonáhle príde jeseň, polárna čiapočka tejto pologule začne opäť rásť. V lete severnej pologuli sa môžu vytvárať mraky, najmä okolo vrcholov sopiek. V iných obdobiach roka môže teplo stúpajúce z trópov, oblasti na oboch stranách rovníka, spôsobiť, že sa v tejto oblasti vytvorí oblačnosť, podobne ako na Zemi. Žiaden prachový oblak na Marse však neprodukuje dážď, hoci je možné, že sa na povrchu planéty môžu dočasne vytvoriť námrazy. Dokonca aj v lete je Mars so svojimi teplotami sotva obyvateľný, s minimami, ktoré dosahujú -140 °C. Občasné maximá však môžu vystúpať až k 20 °C.

\begin{table}[!htbp]
\caption{Dĺžka ročných období na Zemi a na Marse}
\centering
\begin{tabular}{lll}
Ročné obdobie & Dĺžka na Zemi & Dĺžka na Marse  \\
Jar           & 93            & 194             \\
Leto          & 93            & 178             \\
Jeseň         & 90            & 142             \\
Zima          & 89            & 154            
\end{tabular}
\end{table}
Počas obdobia marťanského leta je možné pozorovať úbytok polárnej ľadovej pokrývky, ktorá opäť narastá počas zimy. Práve pod touto vrstvou ľadu sú pravdepodobné ložiská tekutej vody.

Dôležitým parametrom, ktorý určuje dĺžku mesiacov ako na Marse tak aj na Zemi je poloha Slnka na nebeskej sfére pozdĺž ekliptiky z anglického Solar longitude, ktoré sa často skracuje len na Ls. Je to miera polohy telesa obiehajúceho na dráhe okolo Slnka, ktorá sa zvyčajne referenčne začína nulou v okamihu jarnej rovnodennosti. Práve vďaka tomuto parametru sme schopný merať čas tropického roka (rok ročných období) bez chýb, ktoré vychádzaju z kalendárneho určenia roka. Taktiež je to spôsom, ako môžeme opísať rok aj na iných planétach ako Zem, tada aj na Marse.
\begin{figure}[!htbp]
  \centering
  \includegraphics[width=8cm]{img/orbit.png}
  \caption{Uhol naklonenia Marsu voči Slnku v rámci jednej rotácie planéty.}
  \label{solarLS}
  \captionsource{Zdroj: }{http://www-mars.lmd.jussieu.fr/mars/time/solar_longitude.html.}
\end{figure}

\begin{table}[!htbp]
\caption{Rozdelenie dĺžok mesiacov na Marse.}
\label{mesiace_dlzka}
\centering
\begin{tabular}{|l|l|l|l|}
\hline 
Mesiac  & Ls (stupne)   & Sol            & Trvanie (dni)    \\ \hline
1       & 0 - 30        & 0.0 - 61.2     & 61.2             \\ \hline
2       & 30 - 60       & 61.2 - 126.6   & 65.4             \\ \hline
3       & 60 - 90       & 126.6 - 193.3  & 66.7             \\ \hline
4       & 90 - 120      & 193.3 - 257.8  & 64.5             \\ \hline
5       & 120 - 150     & 257.8 - 317.5  & 59.7             \\ \hline
6       & 150 - 180     & 317.5 - 371.9  & 54.4             \\ \hline
7       & 180 - 210     & 371.9 - 421.6  & 49.7             \\ \hline
8       & 210 - 240     & 421.6 - 468.5  & 46.9             \\ \hline
9       & 240 - 270     & 468.5 - 514.6  & 46.1             \\ \hline
10      & 270 - 300     & 514.6 - 562.0  & 47.4             \\ \hline
11      & 300 - 330     & 562.0 - 612.9  & 50.9             \\ \hline
12      & 330 - 360     & 612.9 - 668.6  & 55.7             \\ \hline                                        
\end{tabular}
\end{table}

\begin{table}[!htbp]
\caption{Udalosti na základe uhla naklonenia voči Slnku.}
\centering
\begin{tabular}{|l|l|l|}
\hline 
Ls (stupne) & Udalosť                                       & Sezóna búrok     \\ \hline
0           & Jarná rovnodennosť severnej pologule          & Koniec    \\ \hline
71          & Aphelion (Najväčšia vzdialenosť Mars-Slnko)   &           \\ \hline
90          & Letný slnovrat na severnej pologuli           &           \\ \hline
180         & Jesenná rovnodennosť na severnej pologuli     & Začiatok  \\ \hline
251         & Perihélium (najmenšia vzdialenosť Mars-Slnko) & Trvanie   \\ \hline
270         & Zimný slnovrat na severnej pologuli           & Trvanie   \\ \hline                                      
\end{tabular}
\end{table}

\newpage
\subsection{Predpoveď počasia}
Zem je obalená atmosférou prevažne dusíka, kyslíka a vodnej pary. Vzduch pohybom priestorom nesie so sebou svoje vlastnosti a mení teplotu, vlhkosť, tlak a ostatné iné parametre a teda mení krátkodobý stav atmosféry - počasie. Počasie je teda vo svojej podstate vedľajším produktom našej atmosféry, ktorý prenáša teplo z jedného miesta na druhé.
Ako sme písali v podsekcii \ref{atmosferaZeme}, výkyvy počasia sa menia s lokáciou, kde na Zemi sa práve nachádzame. Taktiež, ako opisujeme v podsekcii \ref{atmosferaMarsu}, atmosféra na planéte Mars je výrazne pokojnejšia a teda zmena počasia je takmer úplne prepojená s ročným obdobím na planéte.
V tejto podsekcii chceme priblížiť vývoj predpovede počasia v histórii meraní.

\paragraph{Prvé zaznamenané merania} sú zaznamenané už v gréckymi filozofmi, ktorí položili základy vednej disciplíny, ktorú dnes nazývame meteorológia. Problémom doby bolo hlavne mylné chápanie prírodných javov, čo spôsobovalo následne zlý výklad vzťahov stavu počasia s rôznymi sprievodnými udalosťami. Pokrok nastal až vynálezom ortuťového barometra Evangelistom Torricellim, talianskym fyzikom a matematikom, ktorý v 17. storočí spôsobil revolúciu v tomto vednom odbore. Takmer palalne k tejto udalosti sa na svetlo sveta dostal aj prvý ortuťový teplomer, ktorý bol vytvorený na základoch objavu Galela Galileiho a jeho vynálezu plynného teplomera, ktorý napriek veľkej nepresnosti a teda nepoužiteľnosti v praxi slúžil ako predloha pre nový koncept s kvapalinou v skle.
17. a 18. storočie patrí medzi najdôležitejšiu etapu v posune meteorológii. Vďaka formulácii fyzikálnych zákonov tlaku plynu, teploty a hustoty od Roberta Boyla a Jacquesa Alexandra Cézara Charlesa, vývoju počtov od Isaaca Newtona a Wilhelma Leibniza, Daltonového zákona o parciálnych tlakoch zmesových plynov a mnohých ďalších zákonov bolo možné lepšie opísať a pochopiť dovtedy neznáme aspekty atmosféry. Následne 19. storočie prinieslo prvé výsledky a teda prvé užitočné predpovede počasia \cite{meteo}.

\paragraph{Analýza synoptických správ o počasí} započala vďaka výmene aktuálneho počasia na mnohých miestach Zeme. Veľmi dôležitým miľníkom bolo zostrojenie telegrafu, čo umožňovalo zhromažďovanie týchto údajov a ich výmenu prakticky po celom svete a tým vytvoriť mapu pozorovaní so vzorcami tlaku, vetra, teploty, oblačnosti a zrážok pri špecifickom čas. Od roku 1849 Joseph Henry vykresľoval denné mapy počasia na základe telegrafických správ a v roku 1869 Cleveland Abbe na Cincinnati Observatory začal poskytovať pravidelné predpovede počasia pomocou údajov získaných telegraficky \cite{meteo}.

\paragraph{Zriaďovanie sietí meteorologických staníc a služieb} viedlo k pravidelnej tvorbe synoptických máp počasia. Bolo to možné po tom, čo boli zorganizované siete staníc merania. Už od roku 1814 dostali pracovníci armádneho sektoru príkaz zaznamenávať údaje o počasí na svojich stanovištiach. Prvé siete meteostaníc začali vznikať prevažne v USA, Veľkej Británii a Holandsku. Postupne sa pridávali všetky veľkomestá po celom svete. Formy získavania dát boli rôzne, od bežných meteorologických staníc po sondy vysielané do atmosféry, aby z niekoľko kilometrovej výšky monitorovali stav \cite{meteo}. 

\paragraph{Moderné trendy a vývoj} technológií poskytlo prostriedky na testovanie nových vedeckých myšlienok. Koncom 20-tych a 30-tych rokov začalo niekoľko skupín vedcov získavať dáta za pomoci rádiovej komunikácie so sondami vo vyššej atmosfére Zeme. Tieto rádiosondy viedli k vzniku pozorovacích sietí vo vyšších častiach atmosféry. Pozorovania teploty a relatívnej vlhkosti pri rôznych tlakoch sú vysielané rádiom späť do stanice, z ktorej sú balóny vypúšťané a sledované radarmi a satelitmi globálneho pozičného systému za účelom zisťovania správania vetra \cite{meteo}.

\paragraph{Počas druhej svetovej vojny} zaznamenala meteorológia významný skok vďaka nasadeniu radara, ktorý poskytoval technológiu na monitorovanie búrok a pozorovanie štruktúr takejto oblačnosti. Moderné radarové systémy využívajú Dopplerov princíp frekvenčného posunu spojeného s pohybom smerom k alebo od radarového vysielača/prijímača na určenie rýchlosti vetra ako aj pohybu búrky \cite{meteo}.

\paragraph{Meteorologické merania zo satelitov a lietadiel} znamenal veľký prelom v meteorologických meraniach. Strednodobé predpovede, ktoré poskytujú informácie na päť až sedem dní vopred, ktoré boli predtým nemožné, začali satelity sprístupňovať globálne. Tieto predpovedné modely boli vyvinuté v americkom Národnom centre pre výskum atmosféry (NCAR), Európskom stredisku pre strednodobé predpovede počasia (ECMWF) a Národnom meteorologickom stredisku USA (NMC) a stali štandardom od 80. rokoch. Meteorologické satelity sa pohybujú po rôznych obežných dráhach a nesú širokú škálu senzorov. Sú dvoch základných typov: nízko letiaci polárny orbiter a geostacionárny orbiter. Kým prvý typ letí vo výške približne 700 kilometrov, druhý typ sa udržuje na orbite vo vzdialenosti približne 36 000 kilometrov. Práve technológia satelitov a celkovo vesmírneho inžinierstva nám umožnila aj tému tejto práce, ktorá pojednáva predikciu meteorologických veličín na inej planéte na základe údajov získavaných či už staticky z roverov na Marse alebo zo satelitov obiehajúcich okolo Marsu.

Poznatky o atmosfére Marsu pôvodne pochádzali z teleskopických záznamov už v roku 1700; ale značne rozkvitla vo veku počítačových modelov a prieskumu kozmických lodí. Vývoj modelov globálnej cirkulácie na Marse má takmer rovnakú históriu ako globálny model Zeme a podobnosti opísané vyššie znamenajú, že mnohé zo základných techník modelovania majú blízke paralely \cite{meteo}.


\paragraph{Poznatky o atmosfére Marsu} pôvodne pochádzali z teleskopických záznamov už v roku 1700. Väčší rozmach prišiel však až vo veku počítačových modelov a kozmickému výskumu. Vývoj modelov globálnej cirkulácie na Marse má takmer rovnakú históriu ako globálny model Zeme a mnohé zo základných techník modelovania majú blízke paralely. Meteorológia na Marse, ktorá sa v rámci slnečnej sústavy najviac podobá Zemi zahŕňa pozorovanie existencie oblakov nesúcich ľad, orografické oblaky tvorené pri pohoriach dôsledkom zdvíhania vzduchu do vyšších polôh alebo západné prúdy, ktoré sa tvoria vo vyšších zemepisných šírkach na oboch pologuliach \cite{meteo}.

Jedným z kľúčových rozdielov medzi Zemou a Marsom je voda. Mnohé trhliny v reliéfe Marsu naznačujú históriu teplejšieho a vlhšieho sveta s tečúcimi riekami a zrážkami a teda aj s omnoho rozmanitejším počasím. Na obrázku \ref{duststorm} je možné vidieť prašnú búrku. Mars zažije každý rok veľa prašných búrok, z ktorých niektoré sa vyvinú do globálnych prašných búrok tak veľkých, že pohltia veľkú časť planéty.

\begin{figure}[!htbp]
  \centering
  \includegraphics[width=8cm]{img/dust_storm_MARS.jpg}
  \caption{Fotka z roku 2018 zachytená Mars Reconnaissance Orbiterom. V spodnej časti obrázku je viditeľná prašná búrka.}
  \label{duststorm}
  \captionsource{Zdroj: }{NASA/JPL-Caltech/Malin Space Science Systems.}
\end{figure}

Obrázok \ref{circLS} nižšie zobrazuje cirkulačný vzor počas rôznych ročných období s červeným tieňovaním zodpovedajúcim pohybu cirkulácie v smere hodinových ručičiek a modrým tieňovaním zodpovedajúcim pohybu proti smeru hodinových ručičiek. 

\begin{figure}[!htbp]
  \centering
  \includegraphics[width=8cm]{img/MARS_circulation_openMars_LS.jpg}
  \caption{Vzory strednej meridionálnej cirkulácie (MMC) v štyroch rôznych ročných obdobiach počas Marsovho roka.}
  \label{circLS}
  \captionsource{Zdroj: }{OpenMARS database.}
\end{figure}

Veľká časť počasia na Marse pripomína púšť na Zemi s opakujúcim sa vzorom teplejších dní a veľmi chladných nocí. Ďaleko od rovníka, najmä na zimnej pologuli, má Mars aj systémy vysokého a nízkeho tlaku, ktoré sú svojou veľkosťou veľmi podobné cyklónom a anticyklónam, ktoré zažívame. Samotný zimný pól je obklopený silným vírom, ktorý môžeme vidieť na obrázku \ref{circPolar} so zimným prúdom, ktorý riadi tieto poveternostné systémy okolo planéty a pohybuje sa zo západu na východ.

\begin{figure}[!htbp]
  \centering
  \includegraphics[width=8cm]{img/MARS_circulation_openMars_polar.png}
  \caption{Polárne projekcie časovo priemernej potenciálnej vírivosti vo výške 25 km nad povrchom nad a. severným a b. južným zimným pólom.}
  \label{circPolar}
  \captionsource{Zdroj: }{OpenMARS database.}
\end{figure}

\subsection{Umelá inteligencia}
Spôsobov, ako definovať umelú inteligenciu je niekoľko. V princípe sa jedná o strojovú simuláciu procesov charakteristických pre ľudí a ich inteligenciu. Cieľom je, aby takto umelo vytvorená inteligencia bola schopná napodobňovať myseľ a činy ľudí. V konečnom dôsledku sa dá definovať umelá inteligencia ako technológia, ktorá stoje učí formou nadobúdania skúseností a prispôsobovaním sa novým vplyvom, prípadne učiť sa vykonávať repetitívne úlohy. V rámci repetitívnosti avšak narážame zároveň aj na definíciu hardvérom ovládanej robotickej automatizácie. Rozdiel spočíva v tom, že umelá inteligencia vykonáva takéto úlohy pomocou rôznych progresívnych algoritmoch učenia, čo spôsobuje jej nezávislosť od neustáleho ľudského zásahu.

Aplikácií pre umelú inteligenciu je neúrekom. Táto technológia môže byť použitá v rôznych sektoroch a odvetviach. Umelú inteligenciu testujeme a používame v zdravotníctve na liečbu u pacientov, predikciu vzniku nádorových ochorení alebo na chirurgické zákroky. Existujú počítače s umelou inteligenciou, ktoré hrajú šach alebo v dnešnej dobe veľmi rýchlo rozvíjajúci sa segment autonómneho riadenia áut. Pri každej z týchto aplikácii musíme zvážiť dôsledky akcie, pretože tieto akcie môžu ovplyvniť, v najhoršom prípade aj ohroziť život ľudí. Kým v šachu je konečným výsledkom víťazstvo v partii, pri autách s vlastným pohonom musí počítačový systém zohľadniť všetky externé údaje a konať tak, aby zabránil zrážke. Ďalším segmentom, kde umelá inteligencia má aplikácie je finančný priemysel, kde sa používa na zisťovanie aktivít v bankovníctve, napríklad nezvyčajné používanie debetných kariet alebo veľké vklady na účty – to všetko pomáha oddeleniu bankových podvodov. Používajú sa aj aplikácie pre umelú inteligenciu, ktoré pomáhajú zefektívniť a zjednodušiť obchodovanie. V neposlednom rade si aplikácie umelej inteligencie nachádzajú miesto aj v predpovediach budúcich udalostí a určovania vývoju na základe vstupov z minulosti. A práve túto vlastnosť chceme v tejto práci využiť a pokúsiť sa využiť aplikáciu predikcie počasia na základe časovej postupnosti meteorologických jednotiek.

Umelá inteligencia môže byť rozdelená do dvoch kategórií: slabá a silná. Slabá umelá inteligencia stelesňuje systém určený na vykonávanie jednej konkrétnej úlohy. Medzi tieto systémy umelej inteligencie patria napríklad hry alebo osobní asistenti. Silné systémy umelej inteligencie sú zase systémy, ktoré vykonávajú úlohy, ktoré sa svojou povahou približujú k tým, ktoré vykonávajú ľudia. Sú naprogramované tak, aby zvládli vyriešit situácie bez nutnosti zásahu ľudského faktora. Tieto druhy systémov možno nájsť v aplikáciách, ako je autonómna mobilita, vesmírny priemysel alebo v nemocničných aplikáciach \cite{ai1}.

Ďalšie rozdelenie umelej inteligencie zahŕňa oblasti vrátane počítačového programovania, strojového učenia, analytických modelov, spracovania prirodzeného jazyka, hlbokého učenia a neurónových sietí \cite{ai2}.

Na vývoj rôznych druhov umelej inteligencie sa píšu pokročilé algoritmy, ktoré sa kombinujú s cieľom rýchlejšej analýzy veľkých objemov údajov. Existuje niekoľko typov algoritmov umelej inteligencie, vrátane:
\paragraph{Regresné algoritmy} sa používajú, keď je cieľovým výstupom spojitá veličina. Počiatočný súbor dát musí mať označenia. Regresné algoritmy spadajú do kategórie učenia s učiteľom.

\paragraph{Klasifikačné algoritmy} prichádzajú do úvahy, ak je potrebné predpovedať výsledok na základe stanoveného počtu fixných, vopred definovaných výsledkov. Príklady zahŕňajú Naive Bayes, the Decision Tree, alebo Logistic Regression.

\paragraph{Algoritmy klastrovania} priraďujú vstupné údaje do dvoch alebo viacerých skupín na základe podobnosti rôznych vlastností. Je to forma strojového učenia bez učiteľa, v ktorej sa algoritmus učí vzory z údajov bez označovania súboru dát.

Rôzne komponenty prispievajú k rôznym typom umelej inteligencie. Medzi prvky umelej inteligencie patria:
\paragraph{Strojové učenie} využíva štatistiku, fyziku a neurónové siete na získavanie poznatkov z údajov bez potreby toho, aby bol systém špecificky naprogramovaný na hľadanie v konkrétnej oblasti.

\paragraph{Neurónové siete} sú systémy strojového učenia, ktoré sa skladajú zo vzájomne prepojených jednotiek - neurónov. Neurónové siete spracovávajú informácie tak, že reagujú na externé vstupy a prenášajú informácie medzi jednotlivými jednotkami. Tento prístup im umožňuje vykonávať viacero prechodov v súbore dát tak, aby našli skryté spojenia.

\paragraph{Hlboké učenie} využíva mohutné neurónové siete s množstvom vrstiev, ktoré s dostatočným výpočtovým výkonom a vylepšenými tréningovými technikami umožňujú systémom identifikovať zložité trendy vo veľkom množstve údajov.

\paragraph{Počítačové videnie} využíva rozpoznávanie vzorov a hlboké učenie na spracovanie, analýzu a pochopenie obrázkov. Systémy dokážu zachytiť obrázky alebo video v reálnom čase a interpretovať ho.

\paragraph{Spracovanie prirodzeného jazyka} umožňuje počítačom analyzovať, porozumieť a vytvárať ľudský jazyk vrátane reči.

\subsection{Strojové učenie}
Strojové učenie je aplikácia umelej inteligencie, ktorá umožňuje systémom učiť sa a zlepšovať sa na základe skúseností bez toho, aby boli explicitne naprogramované. Strojové učenie sa zameriava na vývoj počítačových programov, ktoré majú prístup k dátam a používajú ich na učenie sa. Podobne ako ľudský mozog, ktorý získava vedomosti aj strojové učenie sa pri porozumení entitám spolieha na vstupy, ako sú tréningové údaje alebo znalostné grafy. S definovanými entitami môže začať hlboké učenie.
Proces strojového učenia sa začína pozorovaním alebo údajmi, ako sú príklady, historické údaje, priama skúsenosť alebo inštrukcie. Hľadá vzory v údajoch, aby mohol neskôr vyvodiť závery na základe poskytnutých príkladov. Primárnym cieľom ML je umožniť počítačom učiť sa bez nutnosti ľudského zásahu alebo asistencie a podľa toho prispôsobiť akcie \cite{ml1}.

Termín „strojové učenie“ zaviedol Arthur Samuel, počítačový vedec z IBM a priekopník v oblasti umelej inteligencie. Samuel navrhol počítačový program na hru dámy, ktorý čím viac hral, tým viac sa učil zo skúseností vďaka použitiu algoritmov na vytváranie predpovedí. Ako disciplína strojové učenie skúma analýzu a konštrukciu algoritmov, ktoré sa môžu učiť z dát a predpovedať ich. Vďaka obrovskému množstvu výpočtových schopností za úlohou alebo viacerými špecifickými úlohami možno stroje trénovať na identifikáciu vzorcov a vzťahov medzi vstupnými údajmi a automatizáciu procesov.

Systém učenia algoritmu strojového učenia môžeme rozdeliť na tri hlavné časti:
\begin{itemize}
    \item Rozhodovací proces: Vo všeobecnosti sa na predpovedanie alebo klasifikáciu používajú algoritmy strojového učenia. Na základe vstupných údajov, ktoré môžu byť označené alebo neoznačené algoritmus vytvorí odhad vzoru v dátach.
    \item Chybová funkcia: Chybová funkcia slúži na vyhodnotenie predpovede modelu. Ak existujú známe príklady, chybová funkcia môže vykonať porovnanie na posúdenie presnosti modelu.
    \item Proces optimalizácie modelu: Ak model môže lepšie zodpovedať dátovým bodom v trénovacej množine, potom sa váhy upravia tak, aby sa znížil nesúlad medzi známym príkladom a odhadom modelu. Algoritmus zopakuje tento proces vyhodnocovania a optimalizácie, pričom autonómne aktualizuje váhy, kým sa nedosiahne žiadaný prah presnosti.
\end{itemize}

Metódy strojového učenia môžeme rozdeliť na:
\paragraph{Učenie s učiteľom} je definované používaním označených súborov dát na trénovanie algoritmov, ktoré presne klasifikujú údaje alebo predpovedajú výsledky. Keď sa vstupné údaje vkladajú do modelu, model prispôsobuje svoje hmotnosti, kým nie je model správne prispôsobený. Dochádza k tomu ako súčasť procesu krížovej validácie, ktorý zabezpečí vhodné prispôsobenie modelu. Niektoré metódy používané v kontrolovanom učení zahŕňajú neurónové siete, lineárnu regresiu alebo podporný vektorový stroj (SVM).

\paragraph{Učenie bez učiteľa} využíva algoritmy strojového učenia na analýzu a zoskupovanie neoznačených množín údajov. Tieto algoritmy objavujú skryté vzory alebo zoskupenia údajov bez potreby ľudského zásahu. Jeho schopnosť objavovať podobnosti a rozdiely v informáciách z neho robí ideálne riešenie pre prieskumnú analýzu údajov, stratégie predaja, segmentáciu zákazníkov, rozpoznávanie obrazov a vzorov. Algoritmy používané v učení bez učiteľa zahŕňajú neurónové siete alebo pravdepodobnostné metódy klastrovania.

\paragraph{Čiastočné učenie sa s učiteľom} ponúka balans medzi učením sa s učiteľom a učením sa bez učiteľa. Počas tréningu používa menšiu označenú množinu dát na usmernenie klasifikácie a extrakcie funkcií z väčšej neoznačenej množiny údajov. Učenie s čiastočným dohľadom môže vyriešiť problém nedostatku označených údajov na trénovanie algoritmu učenia s učiteľom.

Najčastejšie je možné nájsť aplikácie strojového učenia v rozpoznávaní reči, počítačovom videní, automatické obchodovanie alebo predikcie časových radov \cite{ai2}.

\subsection{Hlboké učenie}
Hlboké učenie je typ strojového učenia a umelej inteligencie, ktorý napodobňuje spôsob, akým ľudia získavajú určité typy vedomostí. Je dôležitým prvkom vedy o údajoch, ktorý zahŕňa štatistiku a prediktívne modelovanie. Je mimoriadne užitočný pre vedcov, ktorí majú za úlohu zbierať, analyzovať a interpretovať veľké množstvo údajov, nakoľko hlboké učenie robí tento proces rýchlejším a jednoduchším.
Hlboké učenie je možno považovať za spôsob automatizácie prediktívnej analýzy. Zatiaľ čo tradičné algoritmy strojového učenia sú lineárne, algoritmy hlbokého učenia sú usporiadané v hierarchii zvyšujúcej sa zložitosti a abstrakcie.
Aplikácie, ktoré využívajú hlboké učenie, prechádzajú takmer rovnakým procesom, ako človek, ktorý sa učí identifikovať objekt. Každý algoritmus v hierarchii aplikuje nelineárnu transformáciu na svoj vstup a používa to, čo sa naučí, na vytvorenie štatistického modelu ako výstup. Iterácie pokračujú, kým výstup nedosiahne prijateľnú úroveň presnosti. Práve veľký počet vrstiev spracovania údajov je inšpiráciou pre názov týchto typov siete.
V tradičnom strojovom učení je proces učenia sa s učiteľom. Človek, ktorý pripravuje proces musí dbať na správny výber typov vecí, na ktoré sa má počítač zamerať. Pri tomto procese, ktorý sa nazýva extrakcia funkcii úspešnosť závisí od definície sady funkcii, ktoré boli použité. Pri hlbokom učení používame zostavovanie sád funkcii bez učiteľa, teda programátor nemá nad procesom tvorby žiadny dosah. Takýto proces učenia sa ukázal ako rýchlejší a aj presnejší.
Na úvod procesu učenia sa môže použiť trénovacia množina údajov, ktoré vytvoria sadu funkcii a ktoré prispejú k zostaveniu prediktívneho modelu. S každou iteráciou, ktorou model prejde sa stáva nielen zložitejším ale aj presnejším. Na dosiahnutie prijateľne presného modelu je nutné poskytnúť modelu prístup k veľkému množstvu trénovacích údajov.
Na vytvorenie kvalitných modelov hlbokého učenia možno použiť rôzne metódy. Techniky zahŕňajú pokles rýchlosti učenia, prenos učenia alebo vypadnutie \cite{deep_learning}.
\paragraph{Prenos učenia} Tento proces zahŕňa zdokonalenie predtým trénovaného modelu; vyžaduje rozhranie k vnútorným častiam už existujúcej siete. Po vykonaní úprav v sieti je možné vykonávať nové úlohy so špecifickejšími schopnosťami kategorizácie. Pri procese vychádzame z výhody, že nemusíme vychádzať z náhodných váh ale môžeme si pomôcť už vopred natrénovanou sieťou pre podobný problém. Táto metóda má tú výhodu, že vyžaduje oveľa menej údajov ako iné, čím sa skracuje čas výpočtu na minúty alebo hodiny.


Čo sa týka limitujúcich faktorov hlbokého učenia, veľké obmedzenie prichádza z dôvodu, že sa takéto siete učia pozorovaním. Znamená to, že sa učia to, čo vyplýva z údajov, ktoré do siete prichádzajú a na ktorých je sieť trénovaná. V prípade malého množstva alebo vysokej špecifičnosti údajov môžu modely dospieť do štádia trénovania, že ich nebude možné zovšeobecniť.


\subsubsection{Modely hlbokého učenia}
V rámci modelov hlbokého učenia primárne rozdelujeme dve kategórie:
\begin{itemize}
  \item učenie s učitelom alebo učenie pod dohľadom (supervised learning).
  \item učenie bez učitela alebo učenie bez dozoru (unsupervised learning).
\end{itemize}



Najpodstatnejší rozdiel v týchto dvoch prístupov je v tom, ako sú modely trénované. Zatiaľ čo modely učenia s učiteľom sú trénované prostredníctvom príkladov konkrétneho súboru dát, modely bez učiteľa dostávajú iba vstupné údaje a nemajú stanovený výsledok, z ktorého by sa mohli poučiť. Takže výsledok, ktorý sa snažíme predpovedať, nie je v modeli bez učiteľa. V princípe ide o to, že v modeloch bez učiteľa nepoznáme výsledok, ktorý sa snažíme predpovedať, zatiaľ čo modely učenia s učiteľom áno. Pri učení s učiteľom sa zároveň využíva prístup úloh ako je regresia alebo klasifikácia za účelom vytvorenia vzorca, modely bez učiteľa majú učenie založené na zoskupovaní a asociačných pravidlách \cite{deep_learning1}.

Príklady modelov učenia s učiteľom:
\begin{itemize}
  \item klasické neurónové siete: Klasické neurónové siete možno tiež označiť ako viacvrstvové perceptróny. Jeho jedinečná povaha mu umožňuje prispôsobiť sa základným binárnym vzorcom prostredníctvom série vstupov, ktoré simulujú vzorce učenia ľudského mozgu. Viacvrstvový perceptrón je klasický model neurónovej siete pozostávajúci z viac ako 2 vrstiev. Sú používané pri súboroch dát formátovaných do stĺpcov a riadkov alebo klasifikačných a regresných problémov so skutočnými údajmi na vstupe.
  \item konvolučné neurónové siete: Schopnejšia a pokročilejšia variácia klasických umelých neurónových sietí. Je postavená tak, aby zvládla enormné množstvo spracovania a výpočtu údajov. Tento typ sietí boli primárne navrhnuté pre obrazové dáta, napriek tomu dokážu dosiahnuť veľmi dobré výsledky aj s neobrazovými údajmi. Proces budovania siete spočíva v štyroch krokoch: 
  \begin{enumerate}
      \item konvolúcia - proces, pri ktorom sa z našich vstupných údajov vytvárajú mapy objektov.
      \item max-pooling - umožňuje detegovať obrázok, keď je prezentovaný s úpravou.
      \item sploštenie - vloženie údajov do poľa, aby ich sieť mohla prečítať.
      \item plné pripojenie - skrytá vrstva, ktorá počíta stratovú funkciu pre náš model.
  \end{enumerate}
  Primárne využitie tohto typu sietí nájdeme, ako už bolo spomenuté vyššie, pri spracovaní obrazu. Taktiež sú používané pri potrebe komplexných výsledkov na výstupe.
  \item rekurentné neurónové siete - typ umelej neurónovej siete, ktorá využíva sekvenčné údaje alebo údaje z časových radov. Tieto algoritmy hlbokého učenia sa bežne používajú pri bežných alebo dočasných problémoch, ako je preklad jazyka, spracovanie prirodzeného jazyka, rozpoznávanie reči a popisovanie obrázkov. Sú používané v populárnych aplikáciach ako je Siri a iné hlasové vyhľadávania alebo Google Translate. Podobne ako iné typy učenia s učiteľom, aj rekurentné neurónové siete využívajú na učenie trénovacie údaje. Vyznačujú sa svojou „pamäťou“, pretože preberajú informácie z predchádzajúcich vstupov, aby ovplyvnili aktuálny vstup a výstup. Zatiaľ čo tradičné hlboké neurónové siete predpokladajú, že vstupy a výstupy sú na sebe nezávislé, výstup rekurentných neurónových sietí závisí od predchádzajúcich prvkov v sekvencii.
\end{itemize}

Príklady modelov učenia bez učiteľa:
\begin{itemize}
  \item samo organizujúce sa mapy - pracujú s údajmi bez učiteľa a zvyčajne pomáha pri redukcii rozmerov premenných. Výstupná dimenzia je vždy 2-rozmerná pre samo organizujúcu sa mapu. Každá synapsia spájajúca vstupné a výstupné uzly má priradenú váhu. Uzol s najväčším vplyvom v sieti sa nazýva BMU (najlepšia zhodná jednotka) a model aktualizuje svoje váhy, aby sa priblížili k BMU. Sú používané v prípade, že dostupné dáta neobsahujú výstupný parameter, prípadne pri kreatívnych projektoch v rámci umenia.
  \item Boltzmannové stroje - nesledujú trend smerovania. Všetky uzly sú navzájom spojené v kruhovom hyperpriestore. Boltzmannov stroj môže tiež generovať všetky parametre modelu namiesto toho, aby pracoval s pevnými vstupnými parametrami. Takýto model sa označuje ako stochastický a líši sa od väčšiny deterministických modelov. Využíva sa pri monitoringu systémov.
  \item Auto enkodery - Auto kódovače fungujú tak, že automaticky zakódujú dáta na základe vstupných hodnôt, potom vykonajú aktivačnú funkciu a nakoniec dáta dekódujú na výstup. Za pomoci efektu nazývaného "bottleneck", teda určitej podmienke, ktorá separuje premenné vo vstupnej premennej následne prebehne komprimácia dát do kategórii. Následne, ak existuje nejaká inherentná štruktúra, model auto kódovača ju identifikuje a využije na získanie výstupu. Často sa používajú v rámci systémov odporúčaní.
\end{itemize}

\subsubsection{Vrstvy hlbokého učenia}
Nakoľko sa v našej práci zaoberáme predpoveďou počasia, budeme skúmať modely s plne prepojenými a rekurentnými vrstvami, ktoré v nasledujúcej časti aj opíšeme. Vrstva je kontajner, ktorý zvyčajne prijíma vážený vstup, transformuje ho množinou väčšinou nelineárnych funkcií a potom tieto hodnoty odovzdá ako výstup ďalšej vrstve. Vrstva je zvyčajne jednotná, to znamená, že obsahuje iba jeden typ aktivačnej funkcie, združovanie, konvolúciu atď., takže ju možno ľahko porovnávať s inými časťami siete. Prvá a posledná vrstva v sieti sa nazývajú vstupná a výstupná vrstva a všetky vrstvy medzi nimi sa nazývajú skryté vrstvy.

\paragraph{Plne prepojená vrstva} je vrstva, ktorá je hlboko spojená s jej predchádzajúcou vrstvou, čo znamená, že neuróny vrstvy sú spojené s každým neurónom predchádzajúcej vrstvy, ako je to znázornené na obrázku \ref{dense_layer}. Táto vrstva je najčastejšie používanou vrstvou v sieťach umelých neurónových sietí. Funkcia tejto vrstvy je meniť rozmer vektoru pomocou každého neurónu.
\begin{figure}[!htbp]
  \centering
  \includegraphics[width=12cm]{img/dense_layer.png}
  \caption{Názorná ukážka plne prepojenej siete.}
  \label{dense_layer}
  \captionsource{Zdroj: }{https://medium.com/appengine-ai/dense-layers-in-artificial-intelligence-b2f79cc1534a.}
\end{figure}
Neurón tejto vrstvy v modeli dostáva výstup z každého neurónu svojej predchádzajúcej vrstvy, kde prebehlo násobenie maticového vektora. Násobenie maticového vektora je postup, pri ktorom sa riadkový vektor výstupu z predchádzajúcich vrstiev rovná stĺpcovému vektoru vrstvy. Všeobecným pravidlom násobenia matice-vektor je, že riadkový vektor musí mať toľko stĺpcov ako stĺpcový vektor. Následne výpočet prebieha za pomoci vzorca na obrázku \ref{maticavektor}.

\begin{figure}[!htbp]
  \centering
  \includegraphics[width=8cm]{img/dense.png}
  \caption{Všeobecný vzorec násobenia matica-vektor, kde A je matica MxN a x je matica 1xN.}
  \label{maticavektor}
  \captionsource{Zdroj: }{https://analyticsindiamag.com/a-complete-understanding-of-dense-layers-in-neural-networks/}
\end{figure}

Hodnoty matice sú natrénované parametre predchádzajúcich vrstiev, ktoré môžu byť aktualizované pomocou spätnej propagácie. Spätná propagácia je najbežnejšie používaný algoritmus na trénovanie dopredných neurónových sietí. Vo všeobecnosti spätná propagácia v neurónovej sieti počíta gradient stratovej funkcie vzhľadom na váhy siete pre jeden vstup alebo výstup. Výstup pochádzajúci z tejto vrstvy je N-rozmerný vektor \cite{dense}.

\paragraph{LSTM vrstva} má výhodu v procese spracovávania údajov, nakoľko môže v čase zabúdať alebo učiť sa nové vzory. 

\begin{figure}[!htbp]
  \centering
  \includegraphics[width=14cm]{img/lstm(1).png}
  \caption{Princíp LSTM buniek a ich operácie.}
  \label{lstm}
  \captionsource{Zdroj: }{https://towardsdatascience.com/illustrated-guide-to-lstms-and-gru-s-a-step-by-step-explanation-44e9eb85bf21}
\end{figure}

Základným konceptom LSTM je stav bunky a jeho rôzne brány. Stav bunky funguje ako základný parameter, ktorý prenáša informácie až do sekvenčného reťazca. Vytvára sa tým pamäť siete. Stav bunky nesie informácie počas spracovania sekvencie, takže aj informácie z predchádzajúcich krokov sa môžu dostať do neskorších krokov, čím sa znížia účinky krátkodobej pamäte. Keď bunkový stav pokračuje vo svojom procese, informácie sa pridávajú alebo odstraňujú do stavu bunky cez brány. Brány sú rôzne neurónové siete, ktoré rozhodujú o tom, ktoré informácie o stave bunky sú povolené. Brány sa môžu naučiť, aké informácie je dôležité počas tréningu ponechať alebo zabudnúť \cite{lstm}.


Brány obsahuje sigmoidné aktivácie. Aktivácia sigmoidom je podobná aktivácii tanh. Rozdiel je, že kým pri tanh sa hodnoty vložia do rozmedzia hodnôt medzi -1 a 1, pri sigmoide je použitý interval hodnôt medzi 0 a 1. Užitočnosť je vo fakte, že každá zabudnutá hodnota nadobúda hodnotu 0 a teda následné násobenie hodnotu nemení. Taktiež akékoľvek číslo vynásobené 1 sa rovná rovnakej hodnote, preto hodnota zostáva rovnaká alebo je „zachovaná“. Sieť sa v procese učí, ktoré údaje nie sú dôležité a preto môžu byť údaje zabudnuté alebo zachované. Typy brán v sieti sú:
\begin{itemize}
    \item Brána zabudnutia (na obrázku \ref{lstm} fialová farba): Táto brána rozhoduje o tom, aké informácie by sa mali vyhodiť alebo ponechať. Informácie z predchádzajúceho skrytého stavu a informácie z aktuálneho vstupu prechádzajú cez sigmoidnú funkciu. Hodnoty sú medzi 0 a 1. Čím bližšie k 0 znamená zabudnúť a čím bližšie k 1 znamená ponechať.
    \item Vstupná brána (na obrázku \ref{lstm} žltá farba): Na aktualizáciu stavu bunky máme vstupnú bránu. Najprv prenesieme predchádzajúci skrytý stav a aktuálny vstup do sigmoidnej funkcie. To rozhoduje o tom, ktoré hodnoty sa budú aktualizovať transformáciou hodnôt na hodnotu medzi 0 a 1. 0 znamená nedôležité a 1 znamená dôležité. Skrytý stav je tiež možné odovzdať spolu s aktuálnym vstup do funkcie tanh, čo zabezpečí interval hodnôt -1 a 1 čo pomôže s regulováciou siete. Potom sa výstup násobí sigmoidným výstupom. Sigmoidný výstup rozhodne, ktoré informácie je dôležité zachovať z tanhového výstupu.
    \item Stav bunky (na obrázku \ref{lstm} oranžová farba): Teraz by sme mali mať dostatok informácií na výpočet stavu bunky. Stav bunky sa bodovo vynásobí vektorom zabudnutia. Vďaka tomu môžeme vypustiť hodnoty v stave bunky, ktoré sa násobia hodnotami blízkymi 0. Následne vezmeme výstup zo vstupnej brány a urobíme bodové sčítanie, ktoré aktualizuje stav bunky na nové, relevantné hodnoty. Vďaka tomu získavame nový bunkový stav.
    \item Výstupná brána (na obrázku \ref{lstm} modrá farba): Výstupná brána rozhoduje o tom, aký by mal byť ďalší skrytý stav. Skrytý stav obsahuje informácie o predchádzajúcich vstupoch a používa sa aj na predpovede. Najprv sa vloží predchádzajúci skrytý stav a aktuálny vstup do sigmoidnej funkcie. Potom sa odovzdá novo upravený stav bunky funkcii tanh. Tá vynásobí tanh výstup sigmoidným výstupom, aby sa rozhodlo, aké informácie má skrytý stav niesť. Výstupom je skrytý stav. Nový stav bunky a nový skrytý stav sa potom prenesú do ďalšieho časového kroku.
\end{itemize}


Proces, ktorým sa sieť učí je nasledovný:
\begin{enumerate}
    \item Zreťazí sa predchádzajúci skrytý stav a aktuálny vstup.
    \item Táto kombínácia sa dostane do vrstvy zabudnutia. Táto vrstva odstraňuje nerelevantné údaje.
    \item Kandidátska vrstva sa vytvorí pomocou kombinácie. Kandidát obsahuje možné hodnoty na pridanie do stavu bunky.
    \item Kombinácia sa tiež dostáva do vstupnej vrstvy. Táto vrstva rozhoduje o tom, aké údaje z kandidáta sa majú pridať do nového stavu bunky.
    \item Po výpočte vrstvy zabudnutia, kandidátskej vrstvy a vstupnej vrstvy sa stav bunky vypočíta pomocou týchto vektorov a predchádzajúceho stavu bunky.
    \item Vypočíta sa výstup.
    \item Bodovým vynásobením výstupu a nového stavu bunky získame nový skrytý stav.
\end{enumerate}

\paragraph{GRU vrstva} je novšia generácia rekurentných neurónových sietí a je dosť podobná LSTM, preto po tom ako sme opísali fungovanie LSTM vrstvy, opis GRU je omnoho jednoduchší. V GRU sa nepoužíva stav bunky, používa sa len skrytý stav na prenos informácií \cite{gru}. 

\begin{figure}[!htbp]
  \centering
  \includegraphics[width=8cm]{img/gru.png}
  \caption{GRU bunka a jej brány.}
  \label{gru}
  \captionsource{Zdroj: }{https://towardsdatascience.com/illustrated-guide-to-lstms-and-gru-s-a-step-by-step-explanation-44e9eb85bf21}
\end{figure}

Ako môžeme vidieť na obrázku \ref{gru}, koncept obsahuje dve brány, resetovaciu bránu a aktualizačnú bránu.

\begin{itemize}
    \item Aktualizačná brána funguje podobne ako brána zabudnutia a vstupu vo vrstvách LSTM. Rozhoduje, aké informácie vyhodiť a aké nové informácie pridať.
    \item Resetovacia brána je bránou, ktorá sa používa na rozhodnutie, koľko minulých informácií zabudnúť.
\end{itemize}

GRU má menej tenzorových operácií, preto tréningový proces je o niečé rýchlejší ako v prípade LSTM. Neexistuje avšak jednoznačný výsledok, ktorá vrstva je lepšia, preto sa pri vytváraní modelov využívajú oba prístupy a následne sa výsledky porovnávajú za pomoci evaluačných parametrov spomínaných v podsekcii \ref{evaluacia}. 

\subsubsection{Hyperparametre hlbokého učenia}
Pri procese trénovania je nútne nastaviť hyperparametrov za účelom získania dobrých výsledkov. Tie, ktoré najviac ovplyvňujú tréning opisujeme v časti nižšie.
\paragraph{Počet skrytých vrstiev} predstavuje počet vrstiev, ktoré sú medzi vstupnou a výstupnou vrstvou. Mnoho skrytých jednotiek vo vrstve môže vďaka technike regularizácie zvýšiť presnosť. Menší počet jednotiek môže spôsobiť nedostatočné natrénovanie \cite{hyperparams}.

\paragraph{Dropout} je technika regularizácie za účelom zvýšenia presnosti validácie, čím sa zvyšuje sila generalizácie. Vo všeobecnosti sa používa hodnotu v intervale 20 až 50 percent neurónov, pričom 20 percent poskytuje ideálny počiatočný bod. Príliš nízka hodnota má minimálny účinok a príliš vysoká hodnota zase nedostatočné trénovanie. Pravdepodobnosť dosiahnutia lepšieho výkonu nastáva vtedy, keď sa dropout použije vo väčšej sieti. Táto metóda rieši problém preťaženia v sieťach s veľkým množstvom parametrov náhodným vypustením jednotiek a ich spojení z neurónovej siete počas tréningu. Bolo dokázané, že táto metóda zlepšuje výkon neurónových sietí pri úlohách učenia s trénovacou množinou v mnohých oblastiach. 

\paragraph{Activation function}
V neurónových sieťach sa aktivačná funkcia používa na transformáciu vstupných hodnôt neurónov. Do neurónových sietí vnáša nelinearitu , aby sa siete naučili vzťah medzi vstupnými a výstupnými hodnotami.

\paragraph{Rýchlosť učenia} definuje, ako rýchlo sieť aktualizuje svoje parametre. Nízka rýchlosť učenia spomaľuje proces učenia avšak výhodou je plynulá konvergencia. Väčšia rýchlosť učenia urýchľuje učenie, ale nemusí správne konvergovať, ako môžeme vidieť na obrázku \ref{learningrate}. Zvyčajne sa uprednostňuje postupne klesajúca miera rýchlosti učenia. Rýchlosť učenia zároveň riadi, akú veľkú zmenu model zažíva v reakcii na odhadovanú chybu pri každej zmene váh modelu. Metóda poklesu rýchlosti učenia je proces prispôsobenia rýchlosti učenia na zvýšenie výkonu a skrátenie času tréningu.

\begin{figure}[!htbp]
  \centering
  \includegraphics[width=8cm]{img/learningrate.jpg}
  \caption{Gradientný zostup.}
  \label{learningrate}
  \captionsource{Zdroj: }{https://towardsdatascience.com/what-are-hyperparameters-and-how-to-tune-the-hyperparameters-in-a-deep-neural-network-d0604917584a}
\end{figure}

\paragraph{Počet epoch} je počet zobrazení celých tréningových dát sieti počas tréningu. Počet epoch sa zvyšuje, kým validačná presnosť nezačne klesať v porovnaní s presnosťou tréningu, ktorá sa postupne zvyšuje, čo môže spôsobiť pretrénovanie siete.

\paragraph{Veľkosť dávky} je počet čiastkových vzoriek odovzdaných sieti, po ktorých sa vykoná aktualizácia parametrov. Predvolená veľkosť dávky sa väčšinou zadáva v binárnych násobkoch a najčastejšia východzia hodnota predstavuje hodnotu 32. Následne sa veľkosť zvyšuje alebo znižuje podľa procesu tréningu, v prípade pomalého trénovania sa dávka zvyšuje, teda na jednu dávku pripadá viac dát.


\subsection{Výstupy a evaluácia výsledkov pri regresnom probléme}
\label{evaluacia}
Predpoveď počasia je problematika predikcie časového radu, ktorá patrí pod regresné problémy. K regresným problémom sa vo všeobecnosti viažu primárne metriky, ktoré opíšeme v tejto časti.
\paragraph{MAE} Stredná absolútna chyba je metrika hodnotenia modelu používaná v regresných modeloch. Stredná absolútna chyba modelu je priemer absolútnych hodnôt vzhľadom na testovaciu množinu jednotlivých chýb predikcie na všetkých inštanciách testovacej množiny. Každá chyba predpovede je rozdiel medzi skutočnou hodnotou a predpovedanou hodnotou pre inštanciu \cite{mae}. \newline
$MAE = {\frac{1}{n}\sum_{i=1}^{n}(Y-\widehat{Y})}$

\paragraph{MSE} Stredná štvorcová chyba je metrika hodnotenia modelu, ktorá sa často používa pri regresných modeloch. Stredná štvorcová chyba modelu je priemerom štvorcových chýb predikcie vzhľadom na testovaciu množinu vo všetkých prípadoch v testovacej množiny. Chyba predikcie je rozdiel medzi skutočnou hodnotou a predpovedanou hodnotou pre inštanciu \cite{mae}. \newline
$MSE = {\frac{1}{n}\sum_{i=1}^{n}(Y-\widehat{Y})^{2}}$

\paragraph{RMSE} Stredná kvadratická chyba je jedným z najčastejšie používaných meradiel na hodnotenie kvality predpovedí. Ukazuje, ako vzdialene predpovede padajú od skutočných nameraných hodnôt pomocou euklidovskej vzdialenosti. Na výpočet RMSE je nutné vypočítať tzv. reziduum, teda rozdiel medzi predpoveďou a pravdivou hodnotou pre každý dátový bod, vypočítať normu rezidua pre každý dátový bod, vypočítať priemer rezíduí a vypočítať druhú odmocninu tohto priemeru. RMSE sa bežne používa v aplikáciách, kde máme k dispozícii aj reálne dáta, nakoľko RMSE používa a potrebuje skutočné merania v každom predpovedanom údajovom bode. Pri strojovom učení je mimoriadne užitočné mať jedno číslo na posúdenie výkonu modelu, či už počas tréningu alebo validácie. RMSE je jedným z najpoužívanejších parametrov na tento účel \cite{rmse}. \newline
$RMSE = \sqrt{MSE} = \sqrt{\frac{1}{n}\sum_{i=1}^{n}(Y-\widehat{Y})^{2}}$

\newpage

\subsection{Existujúce riešenia}
V rámci štúdia problematiky sme sa sústredili okrem iného aj na projekty, ktoré boli zamerané na Mars a na predpoveď počasia. Podarilo sa nám ich nájsť niekoľko, mnohé sme však nebrali v úvahu z dôvodu duplicitne riešeného problému, respektíve, mnohé z projektov vychádzali z iného projektu a boli v podstate len kópiou. Taktiež sme si všimli, že mnohé projekty sú založené na jednoduchom princípe predpovedania časového radu, teda miesto brania kontextu na danom mieste sa modely učili predikovať jednu premennú výhradne na základe danej premennej a dátumu merania. Napriek faktu, že mnohé výsledky týchto projektov boli prezentované ako vysoko presné, prišli sme k záveru, že tento prístup nie je ten, ktorým by sme sa chceli my zaoberať. Hlavným dôvodom bol fakt, že síce ide o periodicky opakujúci sa rad údajov závislých od času merania, avšak napríklad aj na Marse je možné pozorovať postupné navyšovanie teploty na povrchu. Keďže toto otepľovanie je spôsobné rôznymi javmi na povrchu Marsu, je nutné zahrnúť pri trénovaní viacero závislých premenných, ktoré dokážu presnejšie opísať javy, ktoré sa na Marse dejú a aj budú diať.
Jedným z možných riešení, ktoré je dostupné na internete pojednáva problematiku predikcie meteorologických veličín, konkrétne teploty za pomoci strojového učenia a to rôznymi modelmi. Na testovanie si autori vybrali päť rôznych modelov: CNN, GRU, LSTM, viacvrstvový LSTM a CNN-LSTM. Tieto modely natrénovali a následne sledovali výstupné parametre: RMSE, MSE a MAE. O týchto parametroch budem písať viac v praktickej časti. V princípe, proces predpovede spočíval v príprave dát, zvolení knižníc, premenných a riešení problému s chýbajúcimi dátami a rozdelením dát na časť trénovaciu a časť testovaciu. Následne nastal proces trénovania vyššie spomínanými modelmi, ktoré po natrénovaní vykazovali určité hodnoty parametrov RMSE, MSE a MAE. Z týchto údajov sa dalo usúdiť, že výsledný, ktorý najviac inklinoval k reálnym hodnotám poskytoval model LSTM. Táto informácia iba potvrdila naše domnienky zo štúdii týchto modelov, že práve model LSTM preukazuje pri danom probléme najlepšie výsledky \cite{mars_prediction_whitepaper}. 
Ďalšie príklady zahŕňajú \href{https://www.inta.es/export/sites/default/.galleries/cpess6-Descarga/de-Cabo-Garcia-A_MARSchine-Learning-Prediction-of-Mars-meteorological-variables-using-artificial-neural-networks.pdf}{príklad 1} alebo \href{https://www.dyna-energia.com/search-content-NT/prediction-of-mars-meteorological-variables-using-artificial-neural-networks-2}{príklad 2}. Oba prístupy zvolili podobný spôsob predikcie ako vyššie spomínaný projekt. Opäť sa snažili na základe predchádzajúcich meraní jednej premennej zistiť jej hodnotu v budúcnosti.
V tejto práci sme sa primárne sústredili na cieľ získať model, ktorý pojmä väčšie množstvo parametrov a vyhodnotí jednu konkrétnu premennú, ako sa jej hodnota bude v čase vyvíjať. K tejto problematike existuje niekoľko riešení, avšak tie, ktoré sme našli pojednávali rozdielnu tému, napríklad energetiku. Jedným z najdôležitejších článkov, ktoré sme našli bol tento \href{https://machinelearningmastery.com/multivariate-time-series-forecasting-lstms-keras/}{článok}, ktorý nám načrtol riešenie problematiky viacrozmerných modelov. Práve z tohto článku sme ďalej vychádzali pri finálnom modely viacrozmerného modelu LSTM.

\section{Praktická časť}
V nasledujúcej časti práce si opíšeme proces prípravy súboru dát, výber vhodných technológii, úpravu súboru dát a následne odprezentujeme najskôr čiastočné výsledky, ktoré sa nám podarilo dosiahnuť. Na záver opíšeme výsledky finálneho modelu, ktorý funguje na princípe viackrokového viacvrstvového LSTM modelu.

\subsection{Dáta}
Na úvod sme si pripravili schému prác na diplomovej práci v rámci praktickej časti, ktorú sme následne rozložili do pod sekcii tejto práce. Vďaka tomu sme získali štrukturovaný a prehľadný dokument všetkých procesov, na ktorých sme pracovali a ktoré nás priviedli k výsledným parametrom.

\begin{figure}[!htbp]
  \centering
  \includegraphics[width=8cm]{img/multivariate LSTM Diagram.drawio (1).png}
  \caption{Proces praktickej časti práce.}
  \label{workflow}
  \captionsource{Zdroj: }{draw.io.}
\end{figure}

\subsubsection{Zber súborov dát}
Prvá fáza praktickej časti spočívala v hľadaní historických dát počasia na Marse. Možností bolo niekoľko, nakoľko na orbite okolo Marsu a aj priamo na Marse existuje hneď niekoľko riešení, ktoré monitorujú stav počasia na planéte. Spoločnosť NASA ponúka nástroj k rýchlemu hľadaniu v rámci ich misii, stačí si na  \href{https://pds.nasa.gov/datasearch/data-search/}{stránke} vybrať o akú misiu ide a následne nájdete výsledky relevantných riešení. Medzi tieto riešenia patrí napríklad:
\begin{itemize}
    \item Viking Orbiter 1 \& 2: misia pozostávala z dvoch kozmických lodí, Viking 1 a Viking 2, z ktorých každá pozostávala z orbiteru a pristávacieho modulu. Orbitery mapovali povrch Marsu v rozlíšení 150 až 300 metrov. Niektoré oblasti boli nasnímané s rozlíšením 8 metrov. Pristávacie moduly získali povrchové snímky s vysokým rozlíšením a analyzovali vzorky povrchu na známky života. Meteorologické stanice a seizmometre modulov taktiež dokázali vrátiť údaje na pozemné stanice \cite{viking}.
    \item Mars Reconnaissance Orbiter (MRO): Mars Reconnaissance Orbiter vstúpil na obežnú dráhu Marsu 10. marca 2006. Nasledujúcich šesť mesiacov trvalo skonsolitovanie obežnej dráhy na vedeckú obežnú dráhu 255 x 320 km (s periapsou nad južným pólom a apoapsiou nad severným pólom), čo viedlo k dvanástim slnečno-synchrónnym obehom za deň. Prístroje zahŕňajú kameru zobrazujúcu viditeľné spektrum s vysokým rozlíšením alebo infračervený spektrometer s vysokým rozlíšením na štúdium zloženia povrchu. Vedecké ciele misie sú: charakterizovať súčasnú klímu Marsu a jeho fyzikálne mechanizmy sezónnych a medziročných klimatických zmien, hľadať miesta vykazujúce dôkazy o aktivite vody a identifikovať miesta s najvyšším potenciálom pre vedu na pristátie, prípadne návrat vzoriek budúcimi misiami na Marse \cite{mro}.
    \item InSight: pristál na Marse 26. novembra 2018. Cieľom misie je skúmať planétu do hĺbky, teda smerom do podzemia. Primárna fáza misie je naplánovaná na obdobie jedného marťanského roka. Hlavnými komponentmi prístrojového balíka sú seizmický experiment pre vnútornú štruktúru (SEIS) a sonda tepelného toku a fyzikálnych vlastností (HP3). medzi ďalšie zariadenia patria senzor teploty, vetra a tlaku. Poďla oficialnej stránky aktuálne existujú merania z rozmedzia 30.11.2018 do termínu 29.12.2021 \cite{insight}.
    \item Perseverance: bol vypustený 17. júla 2020 s experimentálnym vrtuľníkom Ingenuity. Pristál v kráteri Jezero (18,4 stupňa severnej zemepisnej šírky, 77,5 stupňa zemepisnej dĺžky.) 18. februára 2021. Kozmická loď na obežnej dráhe odhalila, že kráter obsahoval starovekú deltu a potvrdila prítomnosť ílových minerálov a uhličitanov. Ich vznik predpokladal vlhké prostredie schopné zachovať dôkazy o starovekom živote. Na palube sa nachádza vedecký experiment MEDA (Mars Enviromental Dynamics Analyzer). Cieľom MEDA je poskytovať kontinuálne merania, ktoré charakterizujú denné a sezónne cykly miestnych prašných vlastností a ich časovú odozvu na meteorológiu a lokálne povrchové prostredie \cite{perseverance}. Vzhľadom na aktuálnosť misie ešte nie je možné použiť dostatočný súbor dát na predpoveď počasia. 
    \item Curiosity: primárny prístroj na rovery Curiosity, REMS (Rover Environmental Monitoring Station), zhromažďuje meteorologické údaje. Bol navrhnutý tak, aby zaznamenával šesť atmosférických parametrov: rýchlosť/smer vetra, tlak, relatívnu vlhkosť, teplotu vzduchu, prízemnú teplotu a ultrafialové žiarenie. REMS má frekvenciu merania 1 Hz a typicky odoberá 5-minútové vzorky každú hodinu \cite{curiosity}. Práve súbor dát z tohto zariadenia sme sa rozhodli použiť aj v práci.
\end{itemize}

Zoznam orbiterov, ktorý obiehajú po orbite Marsu môžeme vidieť na obrázku \ref{orbiters}.

\begin{figure}[!htbp]
  \centering
  \includegraphics[width=8cm]{img/orbiters.png}
  \caption{Prehľad orbiterov okolo Marsu a ich určenie.}
  \label{orbiters}
  \captionsource{Zdroj: }{https://atmos.nmsu.edu/data_and_services/atmospheres_data/MARS/mars_orbiter.html.}
\end{figure}
\newpage

Zoznam roverov a statických pristávacích modulov môžeme vidieť na obrázku \ref{landers}

\begin{figure}[!htbp]
  \centering
  \includegraphics[width=8cm]{img/landers.png}
  \caption{Prehľad pristávacích modulov a roverov na povrchu Marsu a ich prístroje.}
  \label{landers}
  \captionsource{Zdroj: }{https://atmos.nmsu.edu/data_and_services/atmospheres_data/MARS/mars_lander.html.}
\end{figure}

V rámci tejto diplomovej práce sme si zvolili využiť súbor dát získavaných z prístroja REMS rovera Curiosity. Ako hlavný dôvod môžeme uviesť dĺžku meraní, ktoré začali 7. augusta 2012 a boli ukončené 6.11.2021. Ide teda o takmer 10 rokov na Zemi alebo po prepočne o približne 5 rokov na Marse. Na obrázku \ref{curiosity_landing_site}, ktorú sme získali vďaka tejto \href{http://www-mars.lmd.jussieu.fr/mars/access.html}{stránke} môžeme vidieť pozíciu pristátia rovera a jeho približnú lokalitu operovania, teda miesto, odkiaľ sme čerpali údaje v rámci planéty Mars. V tabuľke na tejto \href{https://atmos.nmsu.edu/PDS/data/mslrem_1001/DOCUMENT/DATA_EVENTS.TXT}{stránke} je možné taktiež vidieť rôzne udalosti daných dní, ktoré ovplyvnili merania. Podrobné informácie o samotných parametroch je možné nájsť na tejto \href{https://atmos.nmsu.edu/PDS/data/mslrem_1001/DOCUMENT/REMS_RDR_SIS.TXT}{stránke}.
\begin{figure}[!htbp]
  \centering
  \includegraphics[width=14cm]{img/surface_temp.png}
  \caption{Mapa teplôt na povrchu Marsu s vyznačeným miestom meraní Curiosity (zelené označenie v pravej strednej oblasti), konkrétne súradnice: (-4.6 S,137.4 E)}
  \label{curiosity_landing_site}
  \captionsource{Zdroj: }{http://www-mars.lmd.jussieu.fr/}
\end{figure}
\newpage

\subsubsection{Výber vhodných technológií}
Po vyhľadaní a príprave súboru dát sme pristúpili k výberu technológii, ktoré sme v projekte použili. Dôležité bolo vybrať si vhodný programovací jazyk a taktiež priestor, v ktorom bude prebiehať samotné trénovanie.

\paragraph{Python}

\paragraph{Jupyter}



\subsubsection{Úprava súboru dát}
Na obrázku \ref{dataset} môžeme vidieť náhľad finálneho súboru dát pred spracovaním importovaný do systému Jupyter.
\begin{figure}[!htbp]
  \centering
  \includegraphics[width=14cm]{img/datase.png}
  \caption{Náhľad hodnôt v použitom súbore dát.}
  \label{dataset}
  \captionsource{Zdroj: }{colab.google.com}
\end{figure}

Môžeme vidieť dokopy 17 stĺpcov údajov, ktorého parametre si postupne opíšeme:
\begin{itemize}
    \item SOL\char`_CHRONOLOGICAL: parameter predstavujúci chronologické zoradenie dní merania od prvého dátumu merania. Vzhľadom k tomu, že sme do úvahy meraní brali dni od prvého dňa nového roka na Marse, teda pri LS = 0, číslovanie nezačína od hodnoty 0 alebo 1. 
    \item EARTH\char`_DATE: konvertovaná hodnota dátumu merania do pozemského času, resp. dátumu. Dátum je vo formáte MM/DD/YYYY, teda mesiac, deň a rok merania.
    \item LS: jeden z najzásadnejších parametrov v rámci súboru dát. Parameter LS predstavuje uhol naklonenia Marsu v rámci svojej orbity okolo Slnka. V princípe to znamená, že sa meria uhol elipsy od hodnoty 0 stupňov do hodnoty 359 stupňov. Keď sa parameter vráti opať do svojej nulovej pozície, znamená to, že planéta absolvovala jednu rotáciu okolo svojej materskej hviezde a teda prešiel jeden rok na danej planéte. 
    \item MARTIAN\char`_MONTH: Rozdelenie mesiacov tak, ako to môžeme vidieť v tabuľke \ref{mesiace_dlzka}.
    \item MARS\char`_YEAR: Číslo meraného roku na Marse.
    \item T\char`_min: Minimálna denná teplota v Kelvinoch na danej lokácii na Marse.
    \item T\char`_max: Maximálna denná teplota v Kelvinoch na danej lokácii na Marse.
    \item P\char`_min: Minimálny denný tlak na danej lokácii na Marse.
    \item P\char`_max: Maximálny denný tlak na danej lokácii na Marse.
    \item VMR\char`_min: Minimálny denný objemový zmiešavací pomer H2O v rozsahu <PPM (počet častíc na milión) < 400> na danej lokácii na Marse. 
    \item VMR\char`_max: Maximálna denný objemový zmiešavací pomer H2O v rozsahu <PPM (počet častíc na milión) < 400> na danej lokácii na Marse. 
    \item RH\char`_min: Minimálna denná vlhkosť ovzdušia v percentách na danej lokácii na Marse.
    \item RH\char`_max: Maximálna denná vlhkosť ovzdušia v percentách na danej lokácii na Marse.
    \item YEAR\char`_SOL: Chronologicky zoradené dni v danom roku na Marse.
\end{itemize}

\begin{figure}[!htbp]
  \centering
  \includegraphics[width=16cm]{img/full_df.png}
  \caption{Priebeh parametrov za obdobie dvoch rokov na Marse.}
  \label{fulldf}
  \captionsource{Zdroj: }{colab.google.com}
\end{figure}

\begin{figure}[!htbp]
  \centering
  \includegraphics[width=10cm]{img/clean_data.png}
  \caption{Proces čistenia dát.}
  \label{cleandf}
  \captionsource{Zdroj: }{colab.google.com}
\end{figure}

\begin{figure}[!htbp]
  \centering
  \includegraphics[width=16cm]{img/df_desc.png}
  \caption{Opis dát v súbore.}
  \label{dfdesc}
  \captionsource{Zdroj: }{colab.google.com}
\end{figure}

\begin{figure}[!htbp]
  \centering
  \includegraphics[width=16cm]{img/monthly_temp.png}
  \caption{Priebeh teploty v mesiacoch v 2 rokoch na Marse.}
  \label{dfdesc}
  \captionsource{Zdroj: }{colab.google.com}
\end{figure}

\paragraph{Hľadanie závislostí} budem opisovat v tejto casti
\begin{figure}[!htbp]
  \centering
  \includegraphics[width=8cm]{img/heatmap.png}
  \caption{Mapa závislostí medzi parametrami.}
  \label{heatmap}
  \captionsource{Zdroj: }{colab.google.com}
\end{figure}




\newpage

\subsubsection{Výber premennej, ktorú chceme predikovať a závislé premenné}
\subsubsection{Príprava dát na fázu trénovania}

\subsection{Čiastočné výsledky}

\subsubsection{Plne prepojený model} 
\subsubsection{LSTM model} 
\subsubsection{GRU model} 
\subsubsection{Porovnanie plne prepojeného, GRU a LSTM modelu}
\begin{figure}[!htbp]
  \centering
  \includegraphics[width=10cm]{img/GRU, LSTM and Dense.png}
  \caption{Porovnanie výsledkov jednorozmerných modelov.}
  \label{univariate}
\end{figure}

\subsubsection{N-Beats model} 
Tento model bol jedným z prvých, ktorý nás zaujal a chceli sme otestovať jeho kvality. Ide o typ neurónovej siete, ktorý bol prvýkrát opísaný v článku z roku 2019 \cite{n-beats}. Autori uviedli, že N-BEATS prekonali víťaza súťaže o prognózu M4 o 3 \%. M4 je súborom dát dokopy 100 000 časových radov použitých pre súťaž prognostiky Makridakis, výzvy, v ktorej cieľom výskumníkov je vyvíjať stále presnejšie modely predpovedí časových radov. Víťaz M4 bol hybrid medzi rekurentnou neurónovou sieťou a Holt-Wintersovým exponenciálnym vyhladzovaním – zatiaľ čo N-BEATS implementuje „čistú“ hlbokú neurónovú architektúru \cite{m4}. V našom prípade sme použili implementáciu PyTorch N-BEATS knižnice Darts multi-forecast, ktorá je opísaná v článku citácie \cite{beat-impl}. Knižnica kombinuje triedy súvisiace s predpoveďou PyTorch s triedami niekoľkých ďalších balíkov. 

V prvom rade sme sa zamerali na importovanie vhodného súboru dát. Premennú na predikciu sme si zvolili maximálnu dennú teplotu. Následne bolo nutné súbor dát pripraviť špecificky, úplne odlišne od akéhokoľvek iného prístupu, ktorý sme v rámci práce použili. Našimi stĺpcami boli:
\begin{itemize}
    \item series: číslo série poslanej do trénovania.
    \item time\_idx: jedinečné číslo každého merania v sérii.
    \item value: hodnota merania premennej, ktorú chceme predikovať.
    \item static: premenná, ktorú sme v konečnom dôsledku mohli zanedbať, bola pre nás bezvýznamná.
    \item date: dátum merania.
\end{itemize}

Následne sme si určili dĺžku dát, ktoré berieme v úvahu za účelom trénovania a následne koľko dní chceme predikovať. Pre účel trénovania sme zobrali 30 dní z ktorých sme chceli predikovať 10 dní.

\paragraph{Proces tréningu} spočíval v príprave modelu k prijatiu súboru dát a následnom trénovacom procese. V prvom rade išlo o vyjadrenie premenných, ktoré sme vložili do metódy, nastavili sme validačnú množinu, veľkosť dávky a zistili vhodnú hodnotu učiaceho procesu, ako môžeme vidieť aj na obrázku \ref{lr-tune}. 

\begin{figure}[!htbp]
  \centering
  \includegraphics[width=10cm]{img/n-beats-lr.png}
  \caption{Hľadanie ideálneho parametra learning rate.}
  \label{lr-tune}
\end{figure}

\newpage
Po ňom už nasledovalo samostatné trénovanie. Po trénovacom procese sme si dali vykresliť niekoľko príkladov predikcii. Dva príklady môžeme vidieť aj na obrázkoch \ref{beats_res1} a \ref{beats_res2}
\begin{figure}[!htbp]
  \centering
  \includegraphics[width=10cm]{img/beats_res1.png}
  \caption{Výsledok predpovede maximálnej teploty pri strednej chybe 0.92.}
  \label{beats_res1}
\end{figure}
\begin{figure}[!htbp]
  \centering
  \includegraphics[width=10cm]{img/beats_res2.png}
  \caption{Výsledok predpovede maximálnej teploty pri strednej chybe 0.878.}
  \label{beats_res2}
\end{figure}

\newpage
Na záver sme zhodnotili, že dané výsledky sú uspokojivé, nakoľko sledujú trend vývoja a napriek malej dostupnosti dát dokážeme pomocou tejto technológie predpovedať vývoj teploty s presnosťou pod 1 °C. Zároveň sme ale narazili na jeden veľmi dôležitý limitujúci faktor a to ten, že daná technológia nepodporuje pridávanie závislých premenných, teda predpoveď dokáže vytvárať len na základe histórie jednej premennej. To nás viedlo k záveru, že napriek kvalitným výsledkom musíme nájsť model s možnosťou zahrnutia závislých premenných do procesu trénovania. Kód na daný model môžete nájsť v citácii \cite{beats_code}.


\subsubsection{Viacrozmerný model LSTM}
\begin{figure}[!htbp]
  \centering
  \includegraphics[width=10cm]{img/LSTM and stacked LSTM.png}
  \caption{Porovnanie výsledkov jednorozmerného LSTM a jednorozmerného LSTM modelu s viacerými vrstvami.}
  \label{stackedlstm}
\end{figure}


\newpage
\subsection{Finálny model}
Vo finálnom modeli sme využili architektúru viacrozmerného LSTM modelu, ktorý na predpoveď používa údaje za posledných 7 dní aby predpovedal nasledujúci deň. V rámci modelu využívam parametre T\_min, T\_max, P\_min, P\_max, VMR\_min, VMR\_max, RH\_min, RH\_max. Parameter som na základe predchádzajúcich skúseností z trénovanej množiny parametrov vylúčil, nakoľko jeho skresľovanie údajov malo vysoký vplyv na kvalitu predpovede.

\paragraph{Príprava finálneho súboru dát} začal vyčistením všetkých neesenciálnych údajov. Po ubezpečení, že všetky stĺpce súboru sú vo vyhovujúcom formáte sme mohli pokračovať v procese mazania extrémnych dát, ktoré boli s vysokou pravdepodobnosťou chybou merania. 
\begin{figure}[!htbp]
  \centering
  \includegraphics[width=8cm]{img/df_final.png}
  \caption{Súbor dát pripravený na ďalšie spracovanie.}
  \label{dest_abs_error}
\end{figure}
\newpage
V rámci prípravy sme po vyčistení súboru získali dokopy 1297 jedinečných údajov, čo predstavuje približne 2 roky na Marse. Graf vývoja hodnôt môžeme vidieť na nasledujúcom obrázku \ref{features1}.
\begin{figure}[!htbp]
  \centering
  \includegraphics[width=14cm]{img/features1.png}
  \caption{Prehľad vývoja všetkých závislých premenných v rozmedzí 2 rokov.}
  \label{features1}
\end{figure}

\paragraph{Trénovací proces} spočíval v použití minulých hodnôt na predpovedanie budúcnosti. Pre sekvenčné modelovanie sme si zvolili implementáciu rekurentnej neurónovej siete balíka tensorflow s vrstvou LSTM. Vstupom do siete LSTM je 3D pole (vzorky, časové kroky, vlastnosti), pričom:
\begin{itemize}
    \item vzorky — celkový počet sekvencií vytvorených na výcvik.
    \item časové kroky — dĺžka vzoriek.
    \item vlastnosti — počet použitých funkcií.
\end{itemize}

Pred modelovaním je nutné previesť 2D údaje do 3D poľa. Po konverzii získavame pole (1297, 7, 8). Ako môžeme vidieť hodnota vzoriek sa znížila o 8, čo je hodnota tretej dimenzie. Následne môžeme toto pole interpretovať tak, že máme 1297 pozorovaní, každé so 7 riadkami údajov na predikciu a 8 stĺpcami. Existuje o 8 pozorovaní menej vo vzorkách, pretože prvých 7 vzoriek sa vypustí z dôvodu oneskorenia a keďže predpovedáme o 1 krok dopredu, stratí sa aj posledné pozorovanie.

Nasleduje rozdelenie súboru dát na trénovaciu a validačnú množinu. Veľkosť validačnej množiny zároveň určuje počet predikovaných dní. Finálne množiny sú potom opísané na obrázku \ref{mnoziny}.
\begin{figure}[!htbp]
  \centering
  \includegraphics[width=8cm]{img/mnoziny.png}
  \caption{Rozdelenie trénovacej a validačnej množiny.}
  \label{mnoziny}
\end{figure}

Po príprave dát pristupujeme k samotnému trénovaniu modelu, ktorý môžeme vidieť na obrázku \ref{model}. 

\begin{figure}[!htbp]
  \centering
  \includegraphics[width=12cm]{img/model.png}
  \caption{Sumár modelu.}
  \label{model}
\end{figure}


\newpage
\paragraph{Výsledok trénovania} je možné vidieť na obrázku \ref{tabhondot}
\begin{figure}[!htbp]
  \centering
  \includegraphics[width=13cm]{img/tabulka_hodnot.png}
  \caption{Tabuľka zobrazujúca teplotu školovanú, absolútnu a rozdiel hodnôt.}
  \label{tabhondot}
\end{figure}


\begin{figure}[!htbp]
  \centering
  \includegraphics[width=14cm]{img/loss.png}
  \caption{Graf priebehu stratovej funkcie v čase.}
  \label{lossfunction}
\end{figure}

\newline
Následne môžeme dáta vizualizovať vo výsledkoch na obrázku \ref{predpoved}.
\begin{figure}[!htbp]
  \centering
  \includegraphics[width=14cm]{img/predpoved_T_max.png}
  \caption{Graf predpovede parametra Tmax na 129 dní.}
  \label{predpoved}
\end{figure}

\newline
Distribúciu chyby oproti skutočnej hodnote môžeme vidieť na obrázku \ref{dist_error}.
\begin{figure}[!htbp]
  \centering
  \includegraphics[width=14cm]{img/distro_chyby.png}
  \caption{Distribúcia hodnôt chyby v Kelvinoch.}
  \label{dist_error}
\end{figure}

\newline
Hodnotu distribúcie absolútnych chýb je možné vidieť na obrázku \ref{dest_abs_error}
\begin{figure}[!htbp]
  \centering
  \includegraphics[width=14cm]{img/dist_abs_chyby.png}
  \caption{Distribúcia absolútnych chýb v Kelvinoch.}
  \label{dest_abs_error}
\end{figure}

\newpage
Výsledné hodnoty odchýlky voči reálnemu meraniu mám zobrazené v nasledujúcej tabuľke \ref{tab_chyby}.
\newline
\begin{table}[]
\caption{Výsledné parametre hodnôt}
\label{tab_chyby}
\begin{tabular}{ll}
Počet predpovedaných dní: & 129       \\
Stredná hodnota chyby:    & 2.336213 K  \\
Priemerná hodnota chyby:  & 2.224387 K  \\
Minimálna hodnota chyby:  & 0.048517 K  \\
Maximálna hodnota chyby:  & 14.580717 K
\end{tabular}
\end{table}