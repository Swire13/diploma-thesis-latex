% define coloring in language JavaScript
\definecolor{lightgray}{rgb}{.9,.9,.9}
\definecolor{darkgray}{rgb}{.4,.4,.4}
\definecolor{purple}{rgb}{0.65, 0.12, 0.82}
\lstdefinelanguage{JavaScript}{
  keywords={break, case, catch, continue, debugger, default, delete, do, else, false, finally, for, function, if, in, instanceof, new, null, return, switch, this, throw, true, try, typeof, var, void, while, with},
  morecomment=[l]{//},
  morecomment=[s]{/*}{*/},
  morestring=[b]',
  morestring=[b]",
  ndkeywords={class, export, boolean, throw, implements, import, this},
  keywordstyle=\color{blue}\bfseries,
  ndkeywordstyle=\color{darkgray}\bfseries,
  identifierstyle=\color{black},
  commentstyle=\color{purple}\ttfamily,
  stringstyle=\color{red}\ttfamily,
  sensitive=true
}

\section{Motivácia}

\section{Teoretická časť}

\subsection{Počasie všeobecne}
Počasie, v najjednoduchšom ponímaní, predstavuje stav atmosféry na konkrétnej lokácii počas veľmi krátkeho časového intervalu. Zahŕňame do neho také atmosférické javy ako napríklad teplota, vlhkosť, zrážky, ktoré ďalej rozdelujeme podľa druhu a podľa množstva, tlak vzduchi, zloženie ovzdušia, vietor alebo oblačnosť. Často je tento pojem zamieňaný za pojem klíma, ktorý predstavuje informácie o podmienkach z dlhodobého časového horizontu, zväčša 30 rokov.

Počasie najčastejšie definujeme v oblasti zvanej troposféra, teda najnižšej vrstve atmosféry, ktorú budem opisovať v podsekcii \ref{atmosferaZeme}. Tieto atmosférické javy sú značne obmedzené práve na túto vrstvu, nakoľko sa v nej nachádzajú prakticky všetky zrážky a oblačnosť. Jednou z výnimiek sú triskové prúdy, ktoré ovplyvňujú priebeh atmosférického tlaku na hladine mora, čiže sa zapríčiňujú o vplyv na poveternostné podmienky. Počasie ovplyvňuje taktiež geografický reliéf Zeme, nakoľko za pomoci atmosférických javov môže dochádzať k zvetrávaniu a inej degradácii hornín. 

Vo všeobecnosti platí, že premenlivosť počasia sa z rôznou časťou sveta líši. Kým výraznejšie výkyvy počasia môžeme sledovať v stredných pásmach zemepisnej šírky, v tropických oblastiach sa počasie mení len sporaticky. Počasie je závislé aj od mnohých javov, ktoré nazývame anomálie. Ide o jav, ktorý pôsobí na počasie spôsobom, ktorý je veľmi špecifický a často výsledný efekt atmosferických javov je odlišný od očakávania. 

Okrem iného má počasie vplyv aj na osídlenie planéty. Čím lepšie podmienky počasia, tým lepšie podmienky pre pestovanie a celkové bývanie v danej lokalite. V prípade extrémov, akými sú prejavy typu tornádo, krupobitie, snehové búrky, tieto javy môžu ničiť nielen úrodu daného roka a teda spôsobovať problémy so zabezpečením základných surovín, môžu pustošiť aj ľudské obydlia a taktiež môžu ohroziť aj ľudské životy. Kým napríklad v prímorských oblastiach hrozí nebezpečenstvo silných búrok - cyklónov, na pevnine zase môže počasie narobiť problémy vo forme absencie zrážok v akomkoľvek skupenstve alebo silného vetra postupne silnejúcom naprieč dlhými rovinami.

Práve premenlivosť - prakticky hlavná vlastnosť počasia sa podpísala o to, že sa ľudia snažia vytvárať predpovedné modely počasia. Od historických pranostík, ktoré boli postavené na rôznych nevysvetlitelných konštruktoch až po dnešné predpovedanie počasia, ktoré sa vykonáva za pomoci superpočítačov, ktoré sú schopné vytvárať množstvo predpovedných modelov s určitou pravdepodobnosťou z dát získavaných pozemnými stanicami po celom svete a taktiež sieťou meteorologických satelitov. Ide o terabajty dát, ktoré je nutné spracovať za účelom vytvorenia modelu s čo najväčšou pravdepodobnosťou toho, ako bude počasie vyzerať v nasledujúcich dňoch naozaj. Aj tu platí, že kým v stredných pásmach zemepisnej šírky predpoveď počasia je menej presná, v tropických častiach sveta sa počasie dá predpovedať pomerne presne, nakoľko sa mení prevažne periodicky podľa fázy ročného obdobia.

\subsection{Počasie na Zemi}
Zem je treťou planétou v rámci Slnečnej sústavy, zároveň sa nachádza v pásme vhodného pre vznik života. Doba obehu okolo svojej materskej hviezdy trvá 365 dní, rotácia okolo svojej osi trvá 23.93 hodín. Zem je dosiaľ jediné známe teleso s tekutou vodou vhodnou pre život. Až 70 percent zemského povrchu tvorí voda.

Z pohľadu počasia sa Zem vyznačuje komplexnými poveternostnými systémami. Vietor môže dosahovať rýchlosť až 240 km/h vo forme triskových prúdov. Počasie podlieha ešte zložitejším javom z dôvodu interakcie oceánov a iných veľkých vodných plôch ako napríklad morí. Mraky tvorí vodná para zmiešavaná s ľadom. Práve rovnováha týchto dvoch prvkov spôsobuje atmosférický jav nazývaný zrážky. Zrážky môžu vznikať primárne dvomi procesmi. V trópoch, kde sa ľad v oblakoch nevyskytuje ide o spájanie kvapiek vody v oblakoch. Po čase sa kvapky stanú dostatočne ťažké na to, aby opäť padli na povrch Zeme. Druhým spôsobom je interakcia vodnej pary a ľadu severne od oblasti trópov. V tomto prípade ľadové častine majú nižších tlak nasýtených pár ako voda. Postupne sa takéto ľadové kvapôčky spájajú až kým nie je častica natoľko ťažká, aby opäť dopadla na povrch. Vtedy padá na povrch Zeme buď ako dážď alebo vo forme krupobitia.

Výkyvy teplôt sa na Zemi pohybujú od veľmi veľkých v suchom podnebí až po minimálne vo vlhkých oblastiach. V poslednej dobe sa pozornosť upriamuje na skleníkový efekt, ktorý sme ľudskou činnosťou na Zemi dostali až do neželaných hodnôt. Tento efekt však už od počiatku Zeme stál za zrodom života, nakoľko správne množstvo sklenníkových plynov ako napríklad CO2 otepluje atmosféru Zeme na tú teplotu, ktorá sa označuje ako vhodná pre život. Problémom tohto efektu je aktuálne fakt, že teplota vzrastá do hodnôt, ktoré sú pre život neprijateľne z opačného hladiska ako zamŕzanie a teda práve topenie a zvyšovanie globálnej teploty aj lokálnej teploty do hodnôt, ktoré sú pre organizmus nebezpečné.

\subsubsection{Atmosféra Zeme}
\label{atmosferaZeme}
\paragraph{Atmosféra}je plynný obal Zeme, ktorý je pútaný ťiažovou silou k nej a zároveň s ňou aj rotuje. Udávaná horná hranica je 35-40 tisíc kilometrov, avšak samotná vrstva spojito prechádza priamo do kozmického priestoru, preto nie je možné udať presnú hodnotu hranice. Atmosféra je dynamický systém, v ktorom prebiehajú neustále procesy, ktoré sú avšak z dlhodobého hľadiska v relatívnej rovnováhe. Práve v atmosfére sa odohrávajú všetky javy počasia. Taktiež vďačíme práve atmosfére za podmienky vhodné na život na našej planéte, nakoľko zabezpečuje hneď niekoľko dôležitých aspektov:
\begin{itemize}
  \item zjemňuje výkyvy teplôt
  \item chráni organizmy pre slnečným a kozmickým žiarením
  \item chráni povrch pred dopadom menších kozmických telies
  \item zabezpečuje základné podmienky pre život
\end{itemize}
\paragraph{Vývoj atmosféry} na Zemi súvisí s geologcikými a geochemickými procesmi, taktiež aj s existenciou živých organizmov na našej planéte. Súčasná atmosféra Zeme je výsledkom evolúcie, keď pred 4.6 miliardami rokov obsahovala primárne ľahké plyny ako vodík a jeho zlúčeniny (napríklad metán), hélium alebo neón, dnesná atmosféra obsahuje primárne tie plyny, ktoré nereagovali s vodou - oxid uhličitý, dusík a podobne. Atmosféra postupom času chladla, vodná para kondenzovala, čo viedlo k tvorbe oceánov. Oxid uhličitý sa z atmosféry uvolňoval do oceánov a hornín, kde sa viazal v uhličitanoch. Až rozvoj živých organizmov, ktoré v procese fotosyntézy uvolňovali kyslík do ovzdušia, spôsobil jeho hromadenie v atmosfére Zeme. Vďaka tomu sa začala formovať ozónovrstva Zeme, ktorá začala chrániť Zem pred účinkami slnečného ultrafialového žiarenia. Pokles oxidu uhličitého v ovzduší a postupné navyšovanie kyslíka spôsobili pokles skleníkového efektu a markantné zníženie teploty Zeme na úroveň prijateľnú pre život na povrchu. Práve tento krok nenastal, keď sa pozeráme na inú planétu slnečnej sústavy - Venušu. Atmosféra, ktorú poznáme dnes, označujeme ako atmosféra III a registrujeme ju v časovom horizonte asi 400 miliónov rokov. 

\paragraph{Atmosféra Zeme} sa skladá z asi 78 percent dusíka, 21 percent kyslíka, 0,9 percenta argónu a 0,1 percenta iných plynov. Stopové množstvá oxidu uhličitého, metánu, vodnej pary a neónu sú niektoré z ďalších plynov, ktoré tvoria zvyšných 0,1 percenta. Dusík je molekula, ktorá sa do atmosféry dostáva procesom biologického rozkladu alebo sopečnou činnosťou. Zároveň je pre život nevyhnutný, nakoľko sa viaže v bielkovinách. Kyslík sa dostáva do atmosféry fotosyntézou. Je nevyhnutným prvkom, vďaka ktorému môžeme dýchať, taktiež je potrebný na rôzne oxidačné procesy. Argón je plyn, ktorý sa do atmosféry dostáva rádioaktívnym rozpadom draslíka v zemskej kôre. 
Ďalším plyn, ktorý je pre život na našej planéte nevyhnutný je ozón. Vzniká ionizáciou vzduchu napríklad pri búrke alebo taktiež fotochemicky - pôsobením slnečného UV žiarenia. Táto látka, napriek tomu, že je pre živé organizmy jedovatá, vo vyšších polohách atmosféry je pre život klúčová. V rozmedzí 10-50 km sa sústreďuje až 90 percent ozónu, preto tejto časti atmosféry hovoríme aj ozonosféra. Dôvod, prečo je tento plyn jedným zo základným predpokladom života je fakt, že ozón pohlcuje UV žiarenie Slnka, ktoré má škodlivé účinky na živé organizmy. V podsledných desaťročiach sa často spomína pojem ozónová diera, ktorá má za následok úbytok ozónu v atmosfére a to len na niektorých, presne lokalizovaných miestach, väčšinou nad územím severného a južného pólu. Tento jav je pozorovaný približne od 80. rokov minulého storočia a môže zaň najmä freón - uhľovodík, ktorý nahrádza atóm kyslíka za halové prvky ako napríklad flór alebo bróm. Po týchto zisteniach sa upustilo od používania freónov, vďaka čomu nastal proces revitalizácie ozónovej vrstvy, ktorý by mal byť ukončený v roku 2050.

\paragraph{Rozdelenie atmosféry} sa uplatňuje z rôznych hľadísk. Prvým, ktorý opíšem sa určuje podľa zmeny teploty voči výške. Merania dokázali, že teplota s výškou klesá, avšak v niektorých častiach atmosféry teplota narastá. Toto zistenie viedlo k záveru, že atmosféra sa skladá z viacerých vrstiev vzduchu, pričom hlavné vrstvy sú oddelené od seba tenkými prechodnými vrstvami.
\begin{itemize}
  \item Troposféra - výška hranice sa pohybuje od 8 km nad pólmi až po 17 km smerom k rovníku. V troposfére je sustredených približne 80 percent vzduchu a prakticky všetká vodná para, taktiež práve v tomto rozmedzí prebiehajú všetky prírodné javy, ktoré nazývame počasie. Dôležitým pojmom v rámci tejto vrstvy je vertikálny teplotný gradient, ktorý udáva pokles teploty vo vertikálnom smere. Priemerná hodnota sa pohybuje približne v hodnote 0.65 °C na 100 m vertikálnej výšky.
  \item Tropopauza - prechodná vrstva medzi troposférov a stratosférou. Hrúbka tejto hranice predstavuje rádovo niekoľko sto metrov, v záležitosti od podmienok ale može predstavovať až hodnotu do 3 km. Teplotný rozsah sa pohybuje v hodnotách -50 až -80 °C. 
  \item Stratosféra - vrstva oddelená tropopauzou od troposféry siahajúca po výšku 55 km od hladiny mora. V tejto vrstve, narozdiel od troposféry, teplota s narastajúcou vertikálnou výškou stúpa a môže dosiahnuť až 0 °C v najvyšších polohách. Dôvodom sú prebiehajúce fotochemické reakcie sposobené vplyvom slnečného UV žiarenia, konkrétne rozkladom molekuly ozónu, ktorého sa v tejto vrstve nachádza až 90 percent z celkového objemu v atmosfére, preto sa táto oblasť nazýva aj ozónosféra. Koniec tejto vrstvy ohraničuje prechodová vrstva - stratopauza. 
  \item Mezosféra - nachádzajúca sa v rozmedzí 50 - 80 km, pričom teplota s výškou klesá až do -80 °C, zaujímavosťou je, že teplota klesá viac v lete ako v zime. Vo stúpajúcom vertikálnom smere ju následne oddeľuje mezopauza.
  \item Termosféra - vrstva siahajúca až do výšky 300 km, pričom pre posledných 100 km je význačný vzostup teploty s výškou, ktorá na konci predstavuje hodnotu až 1000 °C. V tejto vrstve je vzduch plne ionizovaný, teda obsahuje výhradne elektricky nabité častice. Práve z tejto oblasti je možné pozorovať žiaru meteorov v atmosfére. Taktiež je to oblasť, v ktorej je možné pozorovať polárnu žiaru, nakoľko sú práve v tejto vrstve ideálne podmienky na vstup nabitých častíc zo Slnka. Oddelujúca vrstva - termopauza sa uvádza ako oblasť, kde ešte tento jav môžeme pozorovať. 
  \item Exosféra - finálna časť atmosféry, ktorá už nemá konečné ohraničenie, teda voľne pokračuje do otvoreného kozmu. 
\end{itemize}

Ďalším spôsobom členenia je podľa homogenity vzduchu: 
\begin{itemize}
  \item Homosféra - vertikálny priestor až do výšky 80 km od zemského povrchu. Zaraďujeme sem časti od tropsféry až po hranicu mezopauzi. Charakteristikou je, že v nej nepriebieha obmena zastúpenia zložiek vzduchu.
  \item Heterosféra - s hranicou od mezopauzi ide o časť, v ktorej sa znižuje koncentrácia ľahkých plynov so vzdialenosťou pomalšie ako koncentrácia ťažkých plynov. To je dôvod, prečo vo výške viac ako 1000 km od Zeme prevláda vodík. V tejto oblasti sa taktiež pozoruje priamy vplyv slnečného a kozmického žiarenia, čo spôsobuje napríklad aj jav fotoionizácie, ktorý zapríčiňuje zvýšenú teplotu v týchto oblastiach.
\end{itemize}

Taktiež je možné členenie podľa interakcie so zemským povrchom. Toto členenie je dôležité najmä kvôli zisťovaniu denných meteorologických údajov, pričom rozpoznávame rozdelenie na dve časti:

\begin{itemize}
  \item Hraničná vrstva atmosféry - v tejto časti sa výrazne prejavuje vplyv povrchu Zeme na prejav meteorologických údajov. Ohraničená je výškou, nad ktorou je už prúdenie vzduchu ovplyvnené len tiažovou silou, prípadne rozložením tlaku. Hranica taktiež záleží od reliéfu zemského povrchu, nad ktorou sa nachádza, pri členitejšom reliéfe sa nachádza vyššie ako pri rovinatom teréne. Všeobecne sa ale výška udáva v hodnotácha približne 1500 m. Nad touto úrovňou už nie je teplota vzduchu výrazne závislá od vplyvov zemského povrchu.
  \item Voľná atmosféra - nachádzajúca sa nad hraničnou vrstvou atmosféry. V tejto časti atmosféry už nemajú fyzikálne deje ani meteorologické vplyvy prakticky žiadny efekt.
\end{itemize}
\newline

\subsection{Mars}
Mars je štvrtou a zároveň poslednou planétou vnútornej sústavy planét slnečnej sústavy. Okolo Slnka obieha vo vzdialenosti 227 miliónov kilometrov a čas obehu okolo Slnka je 687 pozemských dní, čo predstavuje dĺžku dňa podobnú našej - 24.6 hodín. Okolo Marsu obiehaju dva malé satelity - Deimos a Phobos, asteroidy, ktoré boli gravitačnou silou Marsu vtiahnuté na jeho obežnú dráhu. Oba satelity sú oproti satelitu Zeme - Mesiacu mnohonásobne menšie s priemermi 11 a 22 metrov. \cite{} Teploty na Marse sú v priemere okolo -63 ˚C. Teploty sa však pohybujú od okolo -140 ˚C v zime na póloch až po 21 ˚C v nižších zemepisných šírkach v lete.

\subsubsection{Atmosféra Marsu}
Za posledné desaťročie bolo na povrch Marsu vypustených mnoho sond za účelom zistenia konzistencie povrchu. Ak by sme chceli prirovnať podmienky na Marse k tým na Zemi, najviac by sa povrch podobal púštným oblastiam. Povrch planéty je posiaty krátermi, ktoré boli vplyvom silného vetra na povrchu postupne uhladené a teda dnes vidíme len malé pozostatky. Práve vietor je jedným z primárnych atmosférických javov na povrchu planéty, nakoľko jeho pôsobením môžu nahromadené prvky podobné piesku vzniesť do ozvdušia a spôsobiť púštne búrky, ktoré sa v rámci roka dokonca na Marse periodicky opakujú. Následne môžeme občas dokonca aj z našej planéty za pomoci silnejšieho ďalekohľadu vidieť výraznejšie začervenanie planéty ako obvykle. Hornina, z ktorej je tento prach na povrchu Marsu zložený pozostáva najmä z danej červenkastej horniny, piesku a pôdy. V niektorých oblastiach Marsu sa vyskytujú miesta so zeleným zafarbením, avšak tento efekt nebol do dnešného dňa vysvetlený. Čo sa týka vody, známe sú ložiská vody pevného skupenstva na póloch planéty. Kvapalnú vodu sa nám do dnešného dňa nepodarilo objaviť napriek dôkazom erodovanej pôdy z plyvu práve vody kvapalného skupenstva. Taktiež sa predpokladá, že voda v kvapalnom skupenstve existuje aj pod povrchom Marsu, avšak vplyvom nízkeho atmosférického tlaku by sa voda vystupujúca na povrch ihneď pretvorila do plynného stavu. 

Atmosféra Marsu sa skladá predovšetkým z oxidu uhličitého. Avšak na rozdiel od Venuše je atmosféra Marsu veľmi tenká, vystavuje planétu bombardovaniu kozmickým žiarením a vytvára veľmi malý skleníkový efekt. Atmosférický tlak na povrchu Marsu je iba 1 až 2 percentá tlaku na Zemi. 

Tak ako na Zemi aj Mars má atmosféru a teda aj počasie. To, ako definujeme tieto dva pojmy na tejto planéte je však mimoriadne odlišné od toho, ako ich vnímame na našej planéte. Ako sme spomínali v sekcii \ref{atmosferaZeme}, Zemskú atmosféru tvorí asi 78 percent dusíka, 21 percent kyslíka, 0,9 percenta argónu a 0,1 percenta iných plynov. Taktiež obsahuje asi 1\% (backslash percent) vodnej pary. Atmosféra Marsu však pozostáva z 95\% (backslash percent) oxidu uhličitého, 3\% (backslash percent) dusíka, 1,6\% (backslash percent) argónu a obsahuje stopy kyslíka, oxidu uhoľnatého, prípadne vody. Vzduch na Marse je veľmi riedky a tlak v porovnaní je taktiež iba 1 až 2 percentá tlaku na Zemi. Pre predstavu, takýto tlak sa vyskytuje vo výške 45 kilometrov našej atmosféry. Tento tlak sa ešte mení s nadmorskou výškou, avšak stále pojednávame o veľmi malých, až zanedbateľných zmenách. Zaujímavosťou v rámci kolísania tlaku je však jeho periodické kolísanie v rámci sezóny. Je to z dôvodu vysokého množstva plynu CO2 v atmosfére, ktorý sa mení s ročnými obdobiami. Vyšší tlak sa vyskytuje počas južných letných mesiacov a najnižší počas severných letných mesiacov. Zmena je spôsobená rozdielom teplôt v danom období. Kým severné polárne zimy sú teplé a krátke, južné polárne zimy sú dlhšie a teplota klesá k nižším hodnotám. Práve južná polárna zima spôsobuje zamŕzanie plynu CO2 priamo v oblastiach južného pólu, čo spôsobuje pokles tohto plynu v atmosfére. Vtedy tlak na Marse klesá o približne 30 percent.

Cirkulácia atmosféry Marsiu je omnoho jednoduchšia ako na Zemi. V nižších zemepisných šírkach je dominantným javom pohyb Hadleyho buniek, ktoré predstavujú stúpajúci zohriaty vzduch v oblasti rovníka. Jedna časť prúdi na sever, druhá na juh do oblasti 30° zemepisnej šírky (do oboch smerov). Tam sa ochladzuje, klesá a prúdi opať k rovníku. V priamom toku zo severu na juh bráni rotácia planéty a reliéf Marsu. Vo vyšších zemepisných šírkach vznikajú polárne vzduchové masy, zároveň sa zo západu na východ tiahne séria oblastí vysokého a nízkeho tlaku. V týchto miestach v čase interakcii s Hadleyho bunkami môžu vznikach fronty počasia pretavujúce sa do búrok. Tie sú však v porovnaní s pozemskými omnoho pokojnejšie.

Veľký rozdiel Zeme a Marsu spočíva v tom, že Mars nemá žiadnu ozónovú vrstvu. Z toho vyplýva, že ultrafialové žiarenie zo Slnka nerušene dosahuje na povrch a teda škodlivo pôsobí na akékoľvek organické zlúčeniny na povrchu. Je to tiež dôvod, prečo atmosféra Marsu nemá žiadnu vrstvu zodpovedajúcej stratosfére Zeme, nakoľko by v takejto vrstve vysokonabité častice nemali s čím interagovať. 

Okrem vyššie spomínaných prašných búrok sa môžu vyskytovať na Marse aj oblaky zložené z ľadových častíc CO2, ktoré sa primárne sústreďujú v oblasti veľkých sopiek, kde sa najčastejšie tvoria vďaka dvíhaním častíc vetrom do vyšších oblastí, kde následne tieto častice kondenzujú. Taktiež sa môžeme stretnúť aj s takzvanými polárnymi kuklami - oblačnosti v polárnej oblasti, ktotér tvoria široké opary. Tento jav zachytilo už niekoľko pozemných roverov, ktoré zaznamenali sneženie počas chladných rán pred východom Slnka. 


\subsubsection{Rok}
\subsubsection{Podmienky}

\begin{center}
\begin{tabular}{llll}
Veličina                                        & Zem     & Mars   & pomer (Mars/Zem) \\
Vzdialenosť od Slnka (10^6 km)                  & 149.6   & 228    & 1.5              \\
Doba obehu (dni)                                & 365.256 & 686.98 & 1.88             \\
Stredná obežná rýchlosť (km/s)                  & 29.78   & 29.1   & 0.81             \\
Inklinácia orbity (stupne)                      & 0       & 1.85   & -                \\
Excentricita                                    & 0.017   & 0.094  & 5.6              \\
Dĺžka rotácie okolo svojej osi (hodiny)         & 23.93   & 24.62  & 1.03             \\
Dĺžka dňa (hodiny)                              & 24      & 24.66  & 1.03             \\
Inklinácia rovníka (stupne)                     & 23.44   & 25.2   & 1.1            
\end{tabular}
\end{center}


\subsubsection{Počasie}
\subsubsection{Obdobia}
\paragraph{Jarná rovnodennosť}
\paragraph{Letný slnovrat}
\paragraph{Jesenná rovnodennosť}
\paragraph{Zimný slnovrat}
\paragraph{Jesenná rovnodennosť}
\paragraph{Zimný slnovrat}
\paragraph{Začiatok sezóny púštnych búrok}
\paragraph{Koniec sezóny púštnych búrok}


\subsection{Možnosti riešenia predpovede}

\subsection{Existujúce riešenia}
Komplexný problém, akým predpoveď počasia nepochybne je by sa mohol na prvý pohľad zdať neriešiteľný z analytického pohľadu. Našťastie, existuje množstvo numerických algoritmov, ktorých cieľom je vyriešiť práve takéto typy problémov. 

Jedným z možných riešení, ktoré je dostupné na internete pojednáva problematiku predikcie meteorologických veličín, konkrétne teploty za pomoci strojového učenia a to rôznymi modelmi. Na testovanie si autori vybrali päť rôznych modelov: CNN, GRU, LSTM, Stacked LSTM, CNN-LSTM. Tieto modely natrénovali a následne sledovali výstupné parametre: RMSE, MSE, MAE, R2. O týchto parametroch budem písať viac v praktickej časti. V princípe, proces predpovede spočínal v príprave dát, zvolení knižníc, zvolení premenných, riešení problému s chýbajúcimi dátami a rozdelením dát na časť trénovaciu a časť testovaciu. Následne nastal proces trénovania vyššie spomínanými modelmi, ktoré po natrénovaní vykazovali určité hodnoty parametrov RMSE, MSE, MAE a R2. Z týchto údajov sa dalo usúdiť, že výsledný, ktoré najviac inklinovali k reálnym hodnotám poskytoval model LSTM. Táto informácia iba potvrdila naše domnienky zo štúdii týchto modelov, že práve model LSTM preukazuje pri danom probléme najlepšie výsledky. 

\subsection{Modely strojového učenia}
\subsubsection{LSTM}
Siete s dlhou krátkodobou pamäťou sú špeciálnym druhom RNN. Boli zavedené, aby sa vyhli problémom s dlhodobou závislosťou. V bežnom RNN sa problém často vyskytuje pri spájaní predchádzajúcich informácií s novými informáciami.

\section{Praktická časť}

\subsection{Dáta}

\subsubsection{Dôležité parametre}
\subsubsection{Príprava dát}

\subsubsection{Implementácia}
\subsubsection{Typy neurónových sieti}
\subsubsection{Odskúšané modely}

\subsection{Výsledky}