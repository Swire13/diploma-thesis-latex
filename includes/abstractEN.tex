Thesis deals with possibilities of using modern Node.js platform in virtual laboratories and create a reference application in combination of other technologies such as Matlab, Simulink, Angular.js and MongoDB. In the introduction we described the characteristics of virtual laboratories and its possible components. We also discussed the possibilities of interaction with the experiment. In the next section we compared existing solutions and their possible lack in nowadays. The following is a section where we have defined the technology and their main characteristics that we planned to use. Implementation of solution was carried out by creating smaller parts. At first we have implemented a simulation motion of projectile in Matlab and Simulink. It was necessary to get data from Simulink to Matlab workspace. Then we had to send them to Node.js using RESTful web services. On the side of Node.js was waiting Socket.io for data receiving that were sent to the web browser. The last part refers to the visualization of data in the browser in the form of graphs, animations, data table and subsequently write data into the database. The result of this thesis is a functional solution called StarkLab where is possible implement own simulation.