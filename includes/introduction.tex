\indent Praktické cvičenie v laboratóriu je dôležitá súčasť pri procese vzdelávania technicky založených ľudí. Ako aj raz povedal starý čínsky filozof Confucius: \textit{"Povedz mi a ja zabudnem, nauč ma a ja si spomeniem, ale nechaj ma zúčastniť sa a ja pochopím."} Zo skúseností už vieme, že človek sa najrýchlejšie učí tak, že si danú vec niekoľko krát sám vyskúša a tak najlepšie pochopí ako to funguje. Nanešťastie nie je možné vždy zabepezpečiť výskumníkom alebo študentom priamy prístup k reálnym zariadeniam pre vykonanie experimentu. Problémov môže byť viacero: vysoká cena vybavenia laboratória, bezpečnosť na pracovisku v závislosti od experimentu prípadne nedostatok kvalifikovaných asistentov.\\
\indent V posledných rokoch sa vývoj virtuálych systémov zvýšil hlavne vďaka technologickej evolúcii softwarového inžinierstva. Pokrok moderných technológií nám dáva solídny základ pri tvorbe, či už všeobecne virtuálných systémov nápomocných pre online vyúčbu, alebo konkrétnych virtuálnych laboratórií, kde sa simulujú fyzikálne javy a procesy. Pri experimentoch vykonávaných vo virtuálnom prostredí je možné zdielať zdroje tohto prostredia na to, aby sa k nemu pripojilo viac užívateľov, ktorí chcú vykonávať rovnaký experiment, čo by pri reálnom zariadení nebolo možné. Vďaka tomu je virtuálne laboratórium vhodným doplnkom štúdia aj výskumu, kde je možné si skúsiť rôzne variácie experimentu bez ohrozenia na zdraví, prípadne zničenia zariadenia a až potom skúšať na reálnom zariadení ak je to potrebné.\\